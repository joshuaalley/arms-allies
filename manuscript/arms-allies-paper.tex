\documentclass[12pt]{article}

\usepackage{fullpage}
\usepackage{graphicx, rotating, booktabs} 
\usepackage{times} 
\usepackage{natbib} 
\usepackage{indentfirst} 
\usepackage{setspace}
\usepackage{grffile} 
\usepackage{hyperref}
\usepackage{adjustbox}
\setcitestyle{aysep{}}


\singlespace
\title{\textbf{Alliance Treaty Design and the Arms-Alliances Tradeoff}}
\author{Joshua Alley\footnote{Graduate Student,
Department of Political Science, Texas A\&M University.}}
\date{{\normalsize \today}}

\bibliographystyle{apsr}

\begin{document}

\maketitle 

\newpage 

\doublespace 

\begin{abstract}



\end{abstract}



\section{Introduction}


How does alliance participation affect military spending? 
Previous scholarship on this issue is divided between two competing camps. 
One group expects alliance participation to reduce military spending. 
The other expects alliance members will spend more on defense. 


In this paper, I address the division between these two perspectives on alliance participation and military expenditures. 
In doing so, I make theoretical and empirical contributions. 
In my argument, I show when alliance participation increases and decreases military spending. 
Differences in the institutional strength of alliance treaties and member size lead alliance participants to employ military spending for different ends. 


Large states use strong treaty commitments to assure their partners. 
Under a weak treaty, junior partners demand additional reassurance, leading large states to sink costs through military spending.
While sunk costs add value to weak commitments, strong alliance treaties are intrinsically valuable. 
Given large foreign policy gains from a strong treaty commitment, small states increase spending to uphold the treaty. 


I test these predictions with a novel research design.
First, I develop a latent measure of alliance treaty strength. 
Then, I employ that measure in a multilevel model which directly compares alliance treaties, and estimates the impact of each treaty on members' military spending.  


% why you should care
Unifying scholarship on alliance participation and military spending has academic and practical value.
Scholarship on this issue has paid little attention to differences between alliances.\footnote{See \citet{DigiuseppePoast2016} for an important exception.} 
As a result, we are left with competing assertions about the characteristics of alliances. 


These arguments are a poor guide for policy discussions. 
Policy debates emphasize reduced spending by alliance members- especially US allies. 
But these debates fail to understand that reduced defense spending by US allies is the result of American efforts to reassure allies by taking costly actions such as deploying troops abroad. 
Those activities require additional military spending, and reflect the proclivity of the US (and other democracies) to form weaker alliance treaty commitments. 
Weak alliance commitments increase the need for costly reassurance, which in turn encourages reduced defense spending. 
Policy discussions emphasize reduced spending by US allies.  


This argument and the evidence I present in this paper has several important implications. 
First, it shows the distributional consequences of alliance treaty design. 
The strength of a commitment shapes how larger and smaller alliance members allocate resources to the military. 


I also show how sunk costs reinforce hands-tying signals in international politics. 
Usually these two actions are considered separately \citep{Fearon1997, FuhrmannSechser2014}, but my argument suggests alliance politics mix the two. 
Rather than substituting sunk costs for hands-tying, sunk costs complement the hands-tying signal of an alliance. 


Last, my argument accentuates important tradeoffs in alliance politics.
While strong commitments may lead junior partners to increase spending, these commitments also come with a risk of entrapment \citep{Benson2012}.
Weak commitments reduce the risk of entrapment, but require additional costs to assure treaty partners they will not be abandoned.


The paper proceeds as follows. 
First, I briefly summarize competing arguments and mixed empirical evidence about alliance participation and military spending. 
Then I describe my argument about treaty strength and member size in more detail. 
The third and fourth sections describe the research design and results. 
The final section concludes with a discussion of the implications for scholarship and policy.  


\section{Force Multiplier vs Foreign Entanglement}

% 2-3 paragraphs per subsection

I divide prior scholarship on alliance participation and military spending into two broad perspectives. 
The foreign entanglement view is that alliance participation will increase military expenditures.
The force multiplier school expects alliance participation to reduce military spending. 


\subsection{Force Multiplier} 


Force multiplier arguments start with the premise that alliances and military spending both provide security.
States substitute between these two foreign policy instruments \citep{MostStarr1989}.  
Alliances provide security that states could not achieve without additional military spending \citep{Morrow1993, Conybeare1994}. 
Because military spending has opportunity costs, states will rely on their allies for security and and reallocate military spending to other desired goods. 


Allied military capability replaces defense expenditures of member states. 
\citet{DigiuseppePoast2016} refine this logic by arguing that states will only reduce spending if their alliance is credible. 
Unreliable alliance capability cannot replace reliable domestic military spending. 


% quick para on public goods model
Another argument in the force multiplier perspective links reduced military spending to a collective action problem. 
\citet{OlsonZeckhauser1966} argue that security from an alliance is a public good, so treaty members provide suboptimal contributions of military spending. 
Each member free-rides on other states, and smaller members exploit the larger. 
Spending less allows alliance members to consume more non-defense goods, but the alliance provides less security. 


Both the substitution and public goods models expect alliance participation will reduce spending. 
These arguments are rooted in the opportunity costs of military spending. 
But the foreign entanglement group argues that alliances provide more than security. 


\subsection{Foreign Entanglement}


The foreign entanglement perspective is less cohesive.
These arguments share a common focus on multiple potential benefits of alliance participation, however. 
Military spending reinforces the benefits of alliance participation. 


% Crap ton of models- one sentance for each. add more detail later if needed. 
\citet{Diehl1994} argues that alliances increase a states foreign policy responsibilities, necessitating extra military spending. 
By expanding what a state can achieve in international relations, states will increase military spending to pursue other foreign policy goals \citep{MorganPalmer2006}. 
\citet{Horowitzetal2017} show that some states increase defense effort to make themselves a more attractive alliance partner. 
Others assert that alliances generate cooperation, leading to higher defense spending \citep{Palmer1990, QuirozFlores2011}
Last, \citet{SeneseVasquez2008} argue that military spending and alliances are part of a spiral towards conflict that leads to simultaneous increases in spending and alliance participation. 


The foreign entanglement perspective contains a crucial insight.
Military spending can complement or facilitate alliance participation. 
However, this perspective does not consider the opportunity costs of military spending. 
Likewise, the force multiplier perspective does not acknowledge synergies between military spending and alliances. 


\subsection{Mixed Evidence} 


Arguments about characteristics of arms and alliances could be settled by a preponderance of empirical evidence. 
Unfortunately, the divided state of theory is reinforced by mixed empirical results.\footnote{Because tests of the public goods model regress military spending as a share of GDP on GDP, I ignore most tests of the public goods theory of alliances in summarizing prior results. These studies are subject to an identification problem.}
Some studies find a positive association between alliance participation and military spending. 
Others find a negative relationship. 


% Specific and general studies
The wide range of methodologies and samples in previous studies can be divided into into specific and general research designs.  
Specific studies examine the impact of a few alliances, usually by tracking how a state responds to the military spending of a key ally. 
General studies compare many states using dummy indicators of alliance participation. 
Each design has different virtues and shortcomings. 


% Virtues and shortcomings- Specific studies of substitution theory of FP
A specific study examines a few alliances in great detail, but lacks generalizability. 
Most support for the substitution of arms and alliances comes from specific designs \citep{BarnettLevy1991, Morrow1993, Sorokin1994, PluemperNeumayer2015}. 
But other specific studies find increased spending by alliance members \citep{ConybeareSandler1990, Chenetal1996}. 


% General models- again, mixed results
General models capture a wide range of state-year observations at the cost of inferences about particular alliances. 
Dummy indicators of alliance participation lump diverse alliances together in a state-level measure. 
\autoref{tab:results-sum} summarizes previous results from general models of alliance participation and military spending. 
Like specific studies, general studies produce mixed results. 
Work by \citet{DigiuseppePoast2016} and \citet{Horowitzetal2017} provides the most reliable estimates. 


\begin{table}[hbt!]
\begin{center}
\begin{tabular}{lccc}
     & Decrease & Increase & Null \\
\hline
\citet{MostSiverson1987} &  &  & X \\
\citet{Conybeare1994} & X & &  \\
\citet{Diehl1994} &  & X &  \\
\citet{Goldsmith2003} &  &  & X \\
\citet{MorganPalmer2006} &  & X & \\ 
\citet{QuirozFlores2011} &  & X &  \\ 
\citet{DigiuseppePoast2016} & X &  & \\ 
\citet{Horowitzetal2017} &  & X & \\ 
\hline
\end{tabular}
\caption{General Findings of Association Between Alliance Participation and Military Spending}
\label{tab:results-sum}
\end{center} 
\end{table}


% Mixed results due to alliance heretogeneity and changes over time. 
Two theoretical and empirical issues explain prior mixed results.
First, there is substantial heterogeneity among alliances.
Treaties vary in their obligations, membership, and capability. 
Alliance heterogeneity makes it difficult to infer general relationships from specific studies, and undermines binary measures of alliance participation in general studies. 
 

Second, alliance members have different roles, depending on their size. 
The public goods theory of alliances suggests that differences in alliance member size matter \citep{OlsonZeckhauser1966, DudleyMontmarquette1981, Garfinkel2004}. 
Large and small alliance participants face different constraints. 
My argument incorporates alliance heterogeneity and differences in member size to explain when alliance participation increases or decreases military spending. 



\section{Argument}



\subsection{Treaty Strength}



\subsection{Alliance Member Size} 




\section{Research Design} 


\subsection{Measuring Alliance Treaty Strength} 



\subsection{Multilevel Model} 



\section{Results}




\section{Discussion}



\section{Conclusion}



%\bibliography{C:/Users/jkalley14/Dropbox/Research/MasterBibliography}  
\bibliography{C:/Users/Josh/Dropbox/Research/MasterBibliography} 





\end{document}
