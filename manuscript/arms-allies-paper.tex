\documentclass[12pt]{article}

\usepackage{fullpage}
\usepackage{graphicx, rotating, booktabs} 
\usepackage{times} 
\usepackage{natbib} 
\usepackage{indentfirst} 
\usepackage{setspace}
\usepackage{grffile} 
\usepackage{hyperref}
\usepackage{adjustbox}
\setcitestyle{aysep{}}


\singlespace
\title{\textbf{Alliance Participation and Military Spending}}
\author{Joshua Alley\footnote{Graduate Student,
Department of Political Science, Texas A\&M University.}}
\date{{\normalsize \today}}

\bibliographystyle{apsr}

\begin{document}

\maketitle 

\newpage 

\doublespace 

\begin{abstract}



\end{abstract}



\section{Introduction}


How does alliance participation affect military spending? 
Previous scholarship on this issue is divided between two competing camps. 
One group expects alliance participation to reduce military spending. 
The other expects alliance members will spend more on defense. 


In this paper, I address the division between these two perspectives on alliance participation and military expenditures. 
In doing so, I make theoretical and empirical contributions. 
In my argument, I show when alliance participation increases and decreases military spending. 
Major and non-major power states use alliances for different purposes, so they respond differently to changes in treaty strength.


Major powers use strong treaty commitments to increase their influence. 
Strong treaties replace military spending as a source of influence, allowing large states to reduce spending. 
Non-major powers emphasize security from alliances, but also have high opportunity costs of military spending.  
Given a strong treaty commitment, these small states cannot reduce military spending without damaging the treaty. 
Weaker treaties still provide some security without tying military support to other costly commitments, allowing non-major powers to reduce military spending. 


I test these predictions with a novel research design.
First, I develop a latent measure of alliance treaty strength. 
Then, I employ that measure in a multilevel model which directly compares alliance treaties, and estimates the impact of each treaty on members' military spending.  


% why you should care
Unifying scholarship on alliance participation and military spending has academic and practical value.
Scholarship on this issue has paid little attention to differences between alliances.\footnote{See \citet{DigiuseppePoast2016} for an important exception.} 
As a result, we are left with competing assertions about the characteristics of alliances. 


These arguments are a poor guide for policy discussions. 
Policy debates emphasize reduced spending by alliance members- especially US allies. 
But these debates fail to understand that reduced defense spending by US allies is the result of different incentives for large and small states. 
Maintaining US influence requires additional military spending, and reflect the proclivity of the US (and other democracies) to form weaker alliance treaty commitments. 
Weak alliance commitments increase the need for costly reassurance, which in turn encourages reduced defense spending. 
Policy discussions emphasize reduced spending by US allies.  


This argument and the evidence I present in this paper has several important implications. 
First, it shows the distributional consequences of alliance treaty design. 
The strength of a commitment shapes how larger and smaller alliance members allocate resources to the military. 


I also show evidence of substitution between sunk costs and hands-tying signals in international politics. 
Usually these two actions are considered separately \citep{Fearon1997, FuhrmannSechser2014}, but alliance politics mix the two. 
Major and non-major powers employ sunk costs and hands-tying in different ways because they have distinct goals and constraints. 


Last, my argument accentuates tradeoffs in alliance politics.
While strong commitments may lead junior partners to increase spending, these commitments also come with a risk of entrapment \citep{Benson2012}.
Weak commitments reduce the risk of entrapment, but require additional spending to maintain influence.


The paper proceeds as follows. 
First, I briefly summarize competing arguments and mixed empirical evidence about alliance participation and military spending. 
Then I describe my argument of treaty strength and member size in more detail. 
The third and fourth sections describe the research design and results. 
The final section concludes with a discussion of the implications for scholarship and policy.  


\section{Force Multiplier vs Foreign Entanglement}

% 2-3 paragraphs per subsection

I divide prior scholarship on alliance participation and military spending into two broad perspectives. 
The foreign entanglement view is that alliance participation will increase military expenditures.
The force multiplier school expects alliance participation to reduce military spending. 


\subsection{Force Multiplier} 


Force multiplier arguments start with the premise that alliances and military spending both provide security.
States substitute between these two foreign policy instruments \citep{MostStarr1989}.  
Alliances provide security that states could not achieve without additional military spending \citep{Morrow1993, Conybeare1994}. 
Because military spending has opportunity costs, states will rely on their allies for security and and reallocate military spending to other desired goods. 


Allied military capability replaces defense expenditures of member states. 
\citet{DigiuseppePoast2016} refine this logic by arguing that states will only reduce spending if their alliance is credible. 
Unreliable alliance capability cannot replace reliable domestic military spending. 


% quick para on public goods model
Another argument in the force multiplier perspective links reduced military spending to a collective action problem. 
\citet{OlsonZeckhauser1966} argue that security from an alliance is a public good, so treaty members provide suboptimal contributions of military spending. 
Each member free-rides on other states, and smaller members exploit the larger. 
Spending less allows alliance members to consume more non-defense goods, but the alliance provides less security. 


Both the substitution and public goods models expect alliance participation will reduce spending. 
These arguments are rooted in the opportunity costs of military spending. 
But the foreign entanglement group argues that alliances provide more than security. 


\subsection{Foreign Entanglement}


The foreign entanglement perspective is less cohesive.
These arguments share a common focus on multiple potential benefits of alliance participation, however. 
Military spending reinforces the benefits of alliance participation. 


% Crap ton of models- one sentance for each. add more detail later if needed. 
\citet{Diehl1994} argues that alliances increase a states foreign policy responsibilities, necessitating extra military spending. 
By expanding what a state can achieve in international relations, states will increase military spending to pursue other foreign policy goals \citep{MorganPalmer2006}. 
\citet{Horowitzetal2017} show that some states increase defense effort to make themselves a more attractive alliance partner. 
Others assert that alliances generate cooperation, leading to higher defense spending \citep{Palmer1990, QuirozFlores2011}
Last, \citet{SeneseVasquez2008} argue that military spending and alliances are part of a spiral towards conflict that leads to simultaneous increases in spending and alliance participation. 


The foreign entanglement perspective contains a crucial insight.
Military spending can complement or facilitate alliance participation. 
However, this perspective does not consider the opportunity costs of military spending. 
Likewise, the force multiplier perspective does not acknowledge synergies between military spending and alliances. 


\subsection{Mixed Evidence} 


Arguments about characteristics of arms and alliances could be settled by a preponderance of empirical evidence. 
Unfortunately, the divided state of theory is reinforced by mixed empirical results.\footnote{Because tests of the public goods model regress military spending as a share of GDP on GDP, I ignore most tests of the public goods theory of alliances in summarizing prior results. These studies are subject to an identification problem.}
Some studies find a positive association between alliance participation and military spending. 
Others find a negative relationship. 


% Specific and general studies
The wide range of methodologies and samples in previous studies can be divided into into specific and general research designs.  
Specific studies examine the impact of a few alliances, usually by tracking how a state responds to the military spending of a key ally. 
General studies compare many states using dummy indicators of alliance participation. 
Each design has different virtues and shortcomings. 


% Virtues and shortcomings- Specific studies of substitution theory of FP
A specific study examines a few alliances in great detail, but lacks generalizability. 
Most support for the substitution of arms and alliances comes from specific designs \citep{BarnettLevy1991, Morrow1993, Sorokin1994, PluemperNeumayer2015}. 
But other specific studies find increased spending by alliance members \citep{ConybeareSandler1990, Chenetal1996}. 


% General models- again, mixed results
General models capture a wide range of state-year observations at the cost of inferences about particular alliances. 
Dummy indicators of alliance participation lump diverse alliances together in a state-level measure. 
\autoref{tab:results-sum} summarizes previous results from general models of alliance participation and military spending. 
Like specific studies, general studies produce mixed results. 
Work by \citet{DigiuseppePoast2016} and \citet{Horowitzetal2017} provides the most reliable estimates. 


\begin{table}[hbt!]
\begin{center}
\begin{tabular}{lccc}
     & Decrease & Increase & Null \\
\hline
\citet{MostSiverson1987} &  &  & X \\
\citet{Conybeare1994} & X & &  \\
\citet{Diehl1994} &  & X &  \\
\citet{Goldsmith2003} &  &  & X \\
\citet{MorganPalmer2006} &  & X & \\ 
\citet{QuirozFlores2011} &  & X &  \\ 
\citet{DigiuseppePoast2016} & X &  & \\ 
\citet{Horowitzetal2017} &  & X & \\ 
\hline
\end{tabular}
\caption{General Findings of Association Between Alliance Participation and Military Spending}
\label{tab:results-sum}
\end{center} 
\end{table}


% Mixed results due to alliance heretogeneity and changes over time. 
Two theoretical and empirical issues explain prior mixed results.
First, there is substantial heterogeneity among alliances.
Treaties vary in their obligations, membership, and capability. 
Alliance heterogeneity makes it difficult to infer general relationships from specific studies, and undermines binary measures of alliance participation in general studies. 
 

Second, depending on their size, alliance members have different goals. 
The public goods theory of alliances suggests that differences in alliance member size matter \citep{OlsonZeckhauser1966, DudleyMontmarquette1981, Garfinkel2004}. 
Large and small alliance participants face different constraints. 
My argument incorporates alliance heterogeneity and differences in member size to explain when alliance participation increases or decreases military spending. 



\section{Argument}

% Focus on growth in spending
This argument predicts growth in military spending. 
Growth in military spending is the best measure of state responses to alliance participation. 
Military spending is subject to a ``ratchet effect'' whereby increases are rarely offset by decreases. 
Using growth in spending instead of changes or levels facilitates comparisons across diverse states and time periods. 
It also limits the risk of spurious inferences from non-stationarity in military spending over time. 


So increases and decreases in military spending refer to growth in military expenditures. 
Lower growth in spending can lead to lower levels of spending, but that is not necessary. 
Alliance participation may accelerate or arrest the growth of defense budgets. 


% Introduce treaty strength, then move to size
Two dimensions shape the association between alliance participation and military spending- state size and alliance treaty strength. 
Major and non-major powers face different opportunities and constraints, so they respond differently to greater alliance treaty strength. 
Strong and weak alliances have different impacts on major and non-major powers.


\subsection{Treaty Strength} 

Alliances are a costly signal of shared interests among members.
Because the treaty is costly, it makes intervention more likely by forming a credible commitment \citep{Fearon1997, Morrow2000}. 
The costs formalized in a treaty commitment give it strength. 


Some treaties are more costly than others. 
Public, formal promises of military support expose alliance participants to audience costs \citep{Morrow2000}.
Other costly commitments generate sunk costs for members, making the commitment more credible \citep{Morrow2000}.
Thus, promises in a treaty provide information to members and potential opponents \citep{Leeds2003}.


Stronger alliance treaties make more costly promises. 
Attaching few conditions to military support is one source of strength \citep{Benson2012}.
Other costly promises include integrated military command, aid, forming international organizations and establishing bases. 
These commitments make strong alliance treaties more credible.


% Both weak and strong alliances provide security
Both strong and weak alliances provide foreign policy gains for members. 
States only form treaties they intend to honor. 
Strong alliances entail a greater loss in freedom of action. 
These treaties mix hands-tying and sunk costs, and the constraints on members' freedom of action make the treaty more credible.
But the same things that make the treaty credible also reduce members' freedom of action.  


% but different tradeoffs
Strong alliance commitments generate distinct tradeoffs for major and minor power participants. 
Alliance participants balance the twin risks of abandonment and entrapment \citep{Snyder1997, Benson2012}. 
Through concerns of abandonment or entrapment and the opportunity costs of military spending, treaty strength shapes incentives of members to increase or decrease military spending.



\subsection{Major Powers}
% Essentially introducing the actors in the theory. 

States are the key actors in this theory. 
However, not all states are equivalent.
Major powers have greater size and foreign policy ambition. 


% Increasing size reduces the opportunity costs of defense spending
Major powers are larger than other states. 
Increasing state size alters the opportunity costs of military spending.  
All states face opportunity costs from military spending, but they are lower in large states.  


As the number of taxpayers falls, the marginal cost per taxpayer of an increase in military spending rises \citep{DudleyMontmarquette1981}. 
Increasing military expenditures impose a larger burden.
Larger economies reduce this tax price of defense effort. 


Major powers also benefit from economies of scale in defense spending. 
More production of defense goods lowers the cost of additional units \citep{Moravcsik1991, AlesinaSpolaore2006}. 
Thus, major powers have lower marginal costs of military spending.  


% 2: increasing scope of FP interests
Major powers also have a wide range of foreign policy interests.
These interests are the result of economic ties, scale, and their ability to pursue a wide range of issues. 
While some states focus on immediate security, others pursue more ambitious foreign policy goals \citep{Fordham2011, MarkowitzFariss2017}. 
Major powers have the means and motivation to pursue broad foreign policy interests.  


%  emphasis on influence. 
Major powers employ alliances and military spending to defend partners and gain influence \citep{Morrow1991}. 
Shaping the policies of other states and ensuring their alignment benefits major powers. 
By aiding other states, major powers increase their influence. 


% How major powers pursue influence
Major powers gain influence by impacting the expected outcome of potential conflicts.\footnote{Influence has many dimensions. Here, influence deals with security.} 
How much influence a major power has depends on how likely they are to intervene, and the amount of capability they possess. 
Intervention by a highly capable state has a large impact on potential war outcomes. 
A state that is seen as highly likely to intervene gains influence, and alliances alter the perceived probability of intervention. 


% Alliances increase probability of intervention
Given shared interests, there is some baseline probability that a state will intervene in conflict. 
By increasing the probability of intervention, alliances give major powers more influence. 
The greater the rise in the perceived probability of intervention, the more influence.


Strong treaties provide more influence by increasing the perceived probability of intervention. 
This allows major powers to realize their desired level of influence without spending as much on military capability. 
Greater treaty strength substitutes for military spending as a source of influence.  


% Focus on influence leads to emphasis on entrapment
Major powers are concerned with entrapment in alliances. 
States can invoke alliance commitments to involve their partners in unwanted conflicts. 
Entrapment results from incentives to uphold a reputation for honoring treaties, and flips the putative direction of influence. 
Strong alliance treaties increase the risk of entrapment \citep{Snyder1997, Benson2012, Yarhi-Miloetal2016}.


% Therefore, tradeoff entrapment and influence. 
Therefore, major powers balance entrapment and influence in alliance treaty design. 
Strong treaties provide more influence, but also come with a risk of entrapment. 
So in some cases, major powers will accept the opportunity costs of higher growth in military spending to retain the freedom of action in a weak treaty. 
Non-major powers face a different tradeoff. 


\subsection{Non-Major Powers} 


% Non-major powers focus on security
Non-major powers emphasize immediate security.
Small states use alliances and military spending to protect their homeland \citep{Morrow1991}. 
In doing so, they face a different set of constraints than major powers. 


% 1: decreasing marginal cost of military spending 
Small states have a higher marginal cost of military spending. 
They are less able to access economies of scale in defense. 
The tax price of spending is also higher than in major powers. 
Thus, non-major powers have higher marginal costs of military expenditures. 
This creates incentives for these states to reduce the defense burden when possible.


% and shift from concern over abandonment to concern of entrapment
Non-major powers fear abandonment--- that their partners will not honor promises of military support.
Potential abandonment generates insecurity. 
Stronger alliance commitments reduce the fear of abandonment. 


% strong treaties and lost autonomy. 
Lost freedom of action is the cost of greater security from a strong treaty for non-major powers.
Though strong treaties provide more security, they also restrict member's freedom of action. 
The influence of other alliance members constrains reductions in defense spending.
Tying promises of military support to other conditions gives partners more leverage to demand adequate defense effort. 
Non-major powers lose some residual control in strong alliances \citep{Lake1996}. 


Under a weak treaty, non-major powers still gain security, but they also retain the freedom to reduce defense spending.    
Given their high opportunity costs of military spending, non-major powers have incentives to rely on allied capability in place of their own. 
Thus, growth in defense spending will increase in alliance treaty strength for non-major powers. 


% Transition to treaty strength 
Their relative emphasis on abandonment or entrapment and different constraints on military spending lead major and non-major powers to respond differently to greater alliance treaty strength. 
Strong treaties will reduce growth in major power military spending, relative to weak treaties. 
Conversely, strong treaties will raise growth in non-major power military spending. 

 


\subsection{Predictions} 

 
% Large states- spending is decreasing in strength
As institutions, alliances structure exchanges among participants \citep{Williamson1985, North1990, DiermeierKrehbiel2003}.
For major powers, strong alliances substitute for military spending as a tool of influence. 
Connecting military support to other promises gives large states more influence.
As a result, increasing alliance treaty strength will reduce growth in military spending in major powers. 


Under a weak treaty, large states have less formal influence. 
But the treaty still increases their foreign policy reach and obligations. 
To maintain their influence, major powers will increase military expenditures given a weaker treaty. 


% Small states- spending in increasing in strength
By contrast, military spending should increase in treaty strength for non-major powers. 
Strong treaties provide more security by adding other costly promises. 
This increases the sense of obligation for alliance partners. 
Moreover, these strong treaties create the expectation members will uphold the treaty. 


Given a weak treaty, non-major powers still gain some security without military support being tied to other obligations. 
As a result, they are free to reduce military spending. 
Allied states less formal leverage to check the incentives of non-major power partners to reduce spending under a weaker treaty. 
Weak treaties provide security, but also give small states the freedom to reduce spending. 
Strong treaties provide more security, with less freedom for small states to reduce spending. 


% here's a 2x2 for the culture  
Therefore, I expect that major and non-major powers will respond differently to participation in strong and weak treaties. 
I summarize the two dimensions of the argument in \autoref{tab:arg-sum}. 
Each cell corresponds to a combination of major power status and treaty strength. 


\begin{table}
\begin{center}
\begin{tabular}{ccc}
      & Strong Treaty      & Weak Treaty  \\
\hline
Major Power & (1)  Decreased Growth Spending   & (2)  Increased Growth Spending        \\
\hline
Minor Power & (3) Increased Growth Spending   & (4) Decreased Growth Spending       \\ 
\hline 
\end{tabular}
\end{center}
\caption{Summary of Argument}
\label{tab:arg-sum}
\end{table}


\autoref{tab:arg-sum} can be distilled into two distinct hypotheses. 
The first prediction addresses growth in military spending for large states as treaty strength increases. 
If weak treaties lead large states to increase spending, and strong treaties decrease spending, then growth in major power military expenditures will decrease as treaty strength increases. 


\begin{quote}
\textsc{Hypothesis 1}: As alliance treaty strength increases, growth in major power military spending will decrease. 
\end{quote}


The second prediction deals with increasing treaty strength in non-major powers. 
If weak treaties lead small states to decrease spending, and strong treaties increase their spending, then growth in non-major power military spending will increase as treaty strength increases. 


\begin{quote}
\textsc{Hypothesis 2}: As alliance treaty strength increases, growth in non-major power military spending will increase. 
\end{quote}


Testing these two predictions requires two things. 
First, my hypotheses compare different alliance treaties, so the the research design should also compare treaties. 
Second, the design must measure alliance treaty strength and compare different treaties.  
The next section describes how I address these two issues. 


\section{Research Design} 


% Need an RD that compares alliances. and a general measure of treaty strength
% Develop latent str. measure and then put it into an ML model


\subsection{Measuring Alliance Treaty Strength} 

% Intuituion behind latent measures: observed char reflects underlying concept



% Justify use



% How the model works


% Show the measure for all alliances- note I'll only focus on treaties w/ military support. 



\subsection{Multilevel Model} 

% Best fit for theoretical process. Can compare alliances. 


% Two separate but connected regressions
% State-level regression- alliances enter through spending matrix.


% Like an interaction between alliance and state-level factors 


% Describe covariates at each level. 


% Because I think the DGP is different for large and small- split sample.


\section{Results}

% show latent strength coefficient in each subset of data



\section{Discussion}



\section{Conclusion}



%\bibliography{C:/Users/jkalley14/Dropbox/Research/MasterBibliography}  
\bibliography{C:/Users/Josh/Dropbox/Research/MasterBibliography} 





\end{document}
