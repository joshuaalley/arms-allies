\documentclass[12pt]{article}

\usepackage{fullpage}
\usepackage{graphicx, rotating, booktabs} 
\usepackage{times} 
\usepackage{natbib} 
\usepackage{indentfirst} 
\usepackage{setspace}
\usepackage{grffile} 
\usepackage{hyperref}
\usepackage{adjustbox}
\usepackage{amsmath}
 \usepackage{multirow} 
\setcitestyle{aysep{}}


\singlespace
\title{\textbf{Alliance Participation, Treaty Depth, and Military Spending}}
\author{Joshua Alley\footnote{Graduate Student,
Department of Political Science, Texas A\&M University.}}
\date{\today}

\bibliographystyle{apsr}

\begin{document}

\maketitle 

\doublespace 

\begin{abstract}
How does alliance participation affect military spending? 
Some argue that alliance membership increases military expenditures, while others contend that it produces spending cuts.
I argue that deep formal defense cooperation modifies the impact of alliance participation on military expenditures.  
Treaty depth reveals a tradeoff between reassurance and greater allied military spending. 
When security-seeking non-major powers join deep alliances they usually decrease military spending, because these treaties are more credible.
Joining shallow alliances often increases non-major power military spending, however.    
I test the argument by creating a measure of alliance treaty depth and employing it in a multilevel model. 
The research design generates new empirical evidence linking alliance participation and percentage changes in state military spending from 1816 to 2007. 
I find that deeper alliance treaties tend to decrease non-major power military spending, and shallow alliances often increase military spending.  
These results help scholars and policymakers better understand a central question about alliance politics that has been debated in scholarship for decades. 
\end{abstract}


\newpage 


\section{Introduction}


Scholars of international relations have long acknowledged that there are two ways for states to increase their security. 
They can invest in indigenous military capability or form alliances \citep{Morgenthau1948, Altfield1984, Morrow1993}.
Because both policies provide security, broadly defined, alliance participation should change how states invest in military capability. 
But exactly how alliances influence military spending remains unclear. 


Existing scholarship contains contradictory theoretical predictions and evidence on the question of alliance participation and military spending. 
One view expects alliance participation will reduce military spending e.g., \citep{BarnettLevy1991, Morrow1993, Conybeare1994}. 
The other predicts alliance participants will spend more on defense e.g., \citep{Diehl1994, MorganPalmer2006, QuirozFlores2011}.
This paper addresses the divide by using alliance treaty design to explain when alliance participation leads to more or less defense spending. 
In doing so, it helps clarify a longstanding debate about alliance politics.


In this paper, I use variation in alliance design and membership to predict how alliance participation affects military spending. 
Scholars have engaged in extensive study of the sources \citep{Mattes2012, Benson2012, Poast2019a} and consequences \citep{Morrow1991, Leeds2003, LeedsAnac2005, Fordham2010, Mattes2012,  Poast2013, Johnsonetal2015} of differences in alliance treaty design and membership. 
Despite the importance of alliance treaty design for outcomes like conflict \citep{Leeds2003, Benson2012} and trade \citep{Long2003, LongLeeds2006} the debate about alliance participation and military spending has largely treated alliances as homogeneous.\footnote{See \citet{DigiuseppePoast2016} for an important exception.}
Given differences in alliance design and membership, alliance participation could plausibly increase or decrease defense expenditures, however. 
In particular, I emphasize how treaty depth modifies the impact of alliance participation on military spending. 
Deep alliances formalize extensive defense cooperation between members. 
In addition to commitments of military support, deep treaties require defense coordination and cooperation among alliance members. 


To explore the consequences of deep and shallow alliances, I examine a particular subset of states--- non-major powers. 
The argument focuses on non-major powers because these states clearly show the tradeoff between reassurance and military spending.\footnote{I explore the process connecting alliance participation and military spending for major powers in a separate paper.} 
Participation in deep alliances allows non-major powers to reduce military spending due to greater treaty credibility and reduced allied leverage on defense spending. 
Joining a shallow alliance often increases non-major power military spending because realizing foreign policy gains from alliance participation depends on defense spending, thanks to concerns with abandonment.

 
I employ a novel research design to test my argument.
First, I develop a latent measure of alliance treaty depth. 
I then incorporate that measure into a multilevel model which estimates how alliance characteristics modify the impact of total allied defense expenditures on percentage changes in military spending.
Allied capability is a useful proxy for alliance participation because it combines the effects of joining an alliance and changing allied capability during treaty membership, both of which change the security benefits of alliance participation. 
Multilevel modeling matches my conditional argument and estimates heterogeneous effects of alliance participation across individual treaties. 
I fit the model on a sample of non-major power states from 1816 to 2007 and find that while deep alliances decrease percentage changes in non-major power military spending, shallow alliances increase spending.


The argument and findings illuminate a salient debate in US foreign policy about the costs and benefits of alliances. 
Advocates of deep engagement \citep{Brooksetal2013} and restraint \citep{Posen2014} in grand strategy have different views of alliances. 
Proponents of restraint argue that the United States should withdraw from many alliances, because allies spend too little on defense, which then increases US defense spending \citep{Preble2009}.
Advocates of continued deep engagement argue that the benefits of alliances exceed the costs and believe that the problem of low allied military spending is overstated \citep{BrandsFeaver2017}. 


My argument and findings suggest that policymakers can adjust alliances to reassure partners or encourage higher allied military spending, but will struggle to do both. 
The United States often forms deep alliances as a way to reassure partners. 
Such efforts to reassure may encourage lower allied military spending, however. 
This tradeoff increases the difficulty of charting a middle course between deep engagement and restraint in grand strategy.  


The paper proceeds as follows. 
First, I summarize competing claims on alliance participation and military spending. 
Then I describe my argument in more detail. 
After the argument, I present the research design and results. 
The final section concludes with a discussion of the results and implications for scholarship and policy.  



\section{Do Alliances Increase or Decrease Military Spending?}


% quick intro and straight into it
Scholarship on alliance participation and military spending is divided between two views.
Each predicts a different average effect of alliance participation by emphasizing one aspect of alliance politics.   


Two types of arguments predict a negative association between alliance participation and defense spending. 
First, the public goods model of \citet{OlsonZeckhauser1966} claims that alliances are subject to a collective action problem because security from an alliance is a public good.
Because alliance security is neither rivalrous nor excludable, members contribute inadequate resources to collective defense. 
Alliance members can ``free-ride'' and smaller states exploit larger partners. 
Lower spending allows alliance members to consume more non-defense goods, but the alliance provides suboptimal security.\footnote{\citet{SandlerForbes1980}, \citet{Oneal1990} and \citet{SandlerHartley2001} all modify the public goods logic while relying on Olson and Zeckhauser's core intuition.} 
Second, substitution arguments recognize that states employ one policy in place of another \citep{MostStarr1989}.
Alliances provide security without requiring additional military spending \citep{Morrow1993, Conybeare1994}. 
Given extra security, states rely on their allies and and reallocate military spending to other goods. 
Both the substitution and public goods models expect that alliance participation reduces military spending due to the opportunity costs of military expenditures. 
States want to rely on their allies for security because higher defense expenditures leave fewer resources for other goods \citep{Fordham1998, Fearon2018}.


A contradictory perspective asserts that alliance participation increases military expenditures. 
Several arguments predict higher military spending by alliance members, using a shared intuition that states increase military spending to support their alliance commitments. 
\citet{Diehl1994} argues that alliances create new foreign policy obligations, necessitating extra military spending.
Because alliances expand what a state can achieve in international relations, states might increase military spending to pursue other foreign policy goals \citep{MorganPalmer2006}.
For example, buffer states use conscription to make themselves a more attractive alliance partner \citep{Horowitzetal2017}.
Others assert that cooperation within alliances generates higher defense spending \citep{Palmer1990, QuirozFlores2011}. 
These predictions of a positive correlation between alliance participation and military spending contradict expectations of lower military spending by alliance members.\footnote{
\citet{SeneseVasquez2008} argue that military spending and alliances are part of a conflict spiral of simultaneous growth in military expenditures and alliance participation, which suggests that conflict behavior drives any correlations between alliances and military spending. 
}


\subsection{Mixed Evidence} 


Debate between the contradictory views of alliances could be settled by a consistent set of results, but mixed findings reinforce the theoretical division.
Some studies find a positive association between alliance participation and military spending. 
Others find a negative relationship.\footnote{
Because tests of the public goods model use military spending as a share of GDP as the their outcome of interest, I do not include most of those results in this summary.} 


% Specific and general studies
Scholars have studied the connection between alliance participation and military spending in two ways. 
General studies of military spending and alliances compare many states through dummy indicators of alliance participation, which collapse alliances into a state-level measure. 
This design compares states with an alliance to those without.
By contrast, specific research designs estimate how states respond to military spending by a few key allies. 
If states reduce their own military spending as allied spending rises, specific studies conclude alliances decrease military spending. 
Many of these studies focus on the North Atlantic Treaty Organization. 


\autoref{tab:results-sum} summarizes previous results findings on the issue of alliance participation and military spending. 
Both specific and general research designs produce mixed findings. 
There is one negative, three positive and two null estimates of the correlation between alliance participation and spending in general studies. 
Specific studies turn up five negative and two positive results.  


\begin{table}[hbt!]
\begin{center}
\begin{tabular}{lcccc}
   Research Design  & Study & Decrease & Increase & Null \\
\hline
\multirow{5}{*}{General} & \citet{MostSiverson1987} &  &  & X \\
 & \citet{Conybeare1994}    & X & &  \\
 & \citet{Diehl1994}        &  & X &  \\
 & \citet{Goldsmith2003}    &  &  & X \\
 & \citet{MorganPalmer2006} &  & X & \\ 
 & \citet{QuirozFlores2011} &  & X &  \\ 
 \hline
 \multirow{7}{*}{Specific} &\citet{ConybeareSandler1990} &   & X &  \\
 &\citet{BarnettLevy1991} & X  &  &  \\
 &\citet{Morrow1993}      & X  &  &  \\
 &\citet{Sorokin1994}     & X  &  &  \\
 &\citet{Chenetal1996}    &  & X &  \\
 &\citet{PluemperNeumayer2015} & X &  &  \\
 &\citet{GeorgeSandler2017} & X &  &  \\
\hline
\end{tabular}
\caption{Findings of the association between alliance participation and military spending. The top block details results from general studies, which compare states with at least one alliance to states without any alliances. The bottom block shows results from specific studies, which examine how national military spending responds to changes in the capability of a few key allies.}
\label{tab:results-sum}
\end{center} 
\end{table}


The mixed empirical results reflect a theoretical problem. 
Both perspectives make unconditional claims about the average effect of alliance participation on military spending.  
With one exception \citep{DigiuseppePoast2016}, scholarship on alliance participation and military spending ignores differences between alliances.
Treaty obligations and membership vary widely across alliances, however \citep{Leedsetal2002}. 
Conflict \citep{Leeds2003, Benson2012} and trade \citep{Long2003, LongLeeds2006} are two issues where alliance design shapes the consequences of treaty participation. 
Building on this work, I focus on a key difference between alliances that can help us understand their heterogeneous effects on military spending: the depth of military cooperation in the treaty.



\section{Argument}

% outline the whole argument
My argument explores how deep military cooperation in an alliance treaty modifies the impact of alliance participation on non-major power military spending.
Given greater treaty credibility in a deep alliance, joining these alliances often reduces non-major power defense spending.
Conversely, fear of abandonment in shallow alliances means non-major power participants in these alliances tend to increase defense expenditures.  


% justify non-major power emphasis
I focus on non-major powers to maintain theoretical and empirical parsimony.
As I explain below, major and non-major powers use alliances to achieve different foreign policy goals, which changes the connection between alliance participation and military spending. 
My argument also provides novel insights about these states.  
Some scholarship and much popular discourse assumes that non-major powers regularly reduce military spending in alliances.
I challenge this assumption by showing that achieving security through alliance participation sometimes requires non-major powers to increase defense expenditures. 


% Summarize the flow of the argument
I start the argument by describing a general framework for non-major power alliances, which treats alliances as institutionalized military cooperation between states. 
Then I discuss how deep formal military cooperation affects alliance credibility. 
Last, I show how alliance depth affects the connection between alliance participation and non-major power military spending. 


\subsection{Cooperation in Alliances}

% opportunistic behavior in alliances and international cooperation 
Alliances are a form of international cooperation. 
Promising military support through a treaty generates a credible commitment of intervention \citep{Fearon1997, Morrow2000}. 
Allied support then helps members achieve crucial foreign policy goals like deterrence or winning wars \citep{Walt1990, Snyder1997}. 
States form alliances so they can use other states' military capabilities \citep{FordhamPoast2014} to advance their foreign policy aims.


Because allied capability gives a treaty foreign policy value, alliance participation is inseparable from allied capability. 
The presence of an alliance treaty formalizes when a state can expect military intervention \citep{Morrow2000}, but the treaty alone does not provide security. 
How helpful an alliance is depends on whether partners provide meaningful military capability. 
Greater allied capability increases the value of an alliance, all else equal \citep{Johnsonetal2015}.\footnote{A binary conceptualization of alliance participation assumes all alliances are equally valuable.}


% Alliance formation models and focus on non-major powers.  
Alliance treaties and the capability they draw on can support many foreign policy aims.
This flexibility often facilitates exchanges between alliance participants. 
One common exchange occurs in asymmetric alliances between major and non-major powers \citep{Morrow1991}. 
Large states form asymmetric alliances to increase their foreign policy influence, while smaller partners gain protection from external threats. 


Not all alliances are asymmetric,\footnote{130 of 289 ATOP alliances with offensive or defensive obligations are asymmetric pacts with at least one major and one non-major power, but a further 122 alliances are symmetric treaties between non-major powers.} but the divergent motives of major and non-major powers reflect general tendencies in alliance politics. 
Major powers often use alliances to address the global balance of power and issues of influence. 
Smaller non-major powers tend to be less ambitious in their alliances and emphasize immediate security. 
As a result, there are distinct processes behind non-major and major power alliance participation and these states respond to allied capability and treaty design in different ways. 


% Non-major power alliances mostly to address external threats
Because non-major powers focus on security, alliances protect them from external threats. 
Treaty depth shapes whether the security benefits of alliance participation depend on military spending. 
Non-major powers can gain increased security with less military spending, even in the face of significant threats, if they form deep alliances. 
Conversely, security gains in shallow alliances may depend on non-major powers maintaining or even increasing their military spending to offset fears of abandonment. 


% Can think about this as an enforcement problem: non-major powers fear abandonment
Given non-major power security concerns, allied support and cooperation is extremely valuable. 
But as with all cooperation, alliance members must account for opportunism, or ``behavior with guile'' \citep{Williamson1985}. 
Even as states commit to an alliance, they can also benefit from defecting and taking advantage of allied cooperation. 
Sometimes the perceived benefits of defection outweigh the long-run benefits of cooperation, so alliance members face an enforcement problem \citep{Fearon1998a, Koremenosetal2001}.


% Problem of abandonment
Non-major powers are especially concerned with abandonment, which is the most common form of opportunism in alliances.
One estimate suggests that the rate of alliance violation is as high as 50\% \citep{BerkemeierFuhrmann2018}.
For security-focused non-major powers, such abandonment is a major threat. 
Concern with abandonment means that alliance members must maintain the perceived credibility of their commitment. 


Abandonment and reduced military spending are related because greater treaty credibility allows states to lower military expenditures.\footnote{The public goods model of alliances calls reduced military spending in alliances free-riding. I do not use this language because it implies reduced defense spending is problematic. But exchange in asymmetric alliances, efficiency gains, or reduced external threats may make reduced spending acceptable.}
Though states can augment the collective military capability of an alliance through their military spending, they can also reduce defense spending and rely on their partners \citep{OlsonZeckhauser1966, Morrow1993, Conybeare1994, SandlerHartley2001}.
Such reductions in military expenditures are more likely under credible alliances. 


As \citet{DigiuseppePoast2016} observe, some alliances have fewer credibility concerns due to members' political regime type.
They show that defense pacts with democracies lower defense spending, as democracies make more credible commitments.
This insight about conditional credibility is a useful starting point because credibility is multifaceted. 
To give three examples, depth, unconditional military support \citep{Benson2012, Chibaetal2015} and issue linkages \citep{LongLeeds2006, Poast2012, Poast2013} are also sources of credibility.\footnote{Though the argument emphasizes depth, the research design accounts for multiple sources of alliance credibility.}


I focus on depth because it provides theoretical leverage to predict when alliance participation increases and decreases military spending, which reveals a trade off between reassurance and defense spending.  
Moreover, treaty depth is a common policy choice.\footnote{In a related paper, I explore the sources of alliance treaty depth. All the sources of depth that paper identifies are included as control variables in the empirical analysis. Average democracy and threat at the time of alliance formation are the two largest sources of treaty depth.} 
Though states probably do not change their political regime to reassure allies, they often form deep alliance treaties. 
Over half of defensive or offensive ATOP alliances have some depth. 
Because deep alliances contain costly commitments, depth reduces the perceived risk of abandonment.  
Costly promises allow alliance members and potential adversaries to infer the credibility of the alliance \citep{Leeds2003, FuhrmannSechser2014}. 


% Here is where to hit the substantive point
Credible alliances offer an opportunity for non-major powers to reduce military spending. 
What are allies preferences over military spending?
Do allies always prefer that non-major powers increase military spending? 
Reduced defense spending by non-major powers is not always problematic, especially if the alliance incorporates exchange. 
But there are circumstances where states would prefer their allies spend more on the military. 


Regardless of allied preferences, greater alliance depth and credibility are unlikely to increase non-major power military spending. 
Enforcing higher allied defense effort is difficult.
Normative appeals to common interests are ineffective. 
Though verbal communication or ``cheap talk'' has value in international politics \citep{Trager2010}, it is unlikely to overcome the opportunity costs of defense spending. 
Even after reducing defense expenditures, alliance members retain foreign policy benefits and can reallocate resources to other priorities. 
The ability to reduce defense spending and spend more on other goods sometimes motivates states to form alliances \citep{Kimball2010, AllenDigiuseppe2013}. 


% Need leverage 
Alliance members must have leverage to encourage increases in allied defense spending. 
Leverage either comes from a credible threat to abandon states who spend too little or control over allied policies. 
Policy control of allied spending decisions occurs when the alliance reflects hierarchical relationships \citep{Lake1996}. 
Without such direct influence, states must possess a credible threat to leave the alliance over low defense spending. 
Otherwise, allies will dismiss weaker signals and threats due to uncertainty and incomplete information. 


% adding credibility/reassuring reduces leverage
Reassuring allies reduces the credibility of threats to abandon states that spend too little on defense. 
States cannot simultaneously reassure their allies and maximize leverage on defense spending. 
As alliance members use costly commitments to reassure, partners can reduce defense spending. 


% Talk about more credible alliances
In less credible treaties, such as alliances between erstwhile rivals \citep{NiouZeigler2019}, failure to spend on the military could increase the risk of abandonment, so members have less freedom to reduce defense spending. 
Moreover, although states in less credible alliances increase their foreign obligations, they also face a greater risk of abandonment. 
As treaty credibility falls, the foreign policy benefits of alliance participation become contingent on military spending because alliance members hedge against abandonment and partners can use that concern as leverage.
Thus, alliances with less credibility will tend to increase military spending. 


% Transition- depth of military cooperation shows this tradeoff 
Treaty depth highlights this tradeoff between reassurance and military spending.
Where deep alliances often reduce non-major power military spending, participation in shallow alliances often increases it. 
Stipulating deep cooperation reassures partners and reduces leverage to check reductions in military expenditures. 
Credibility from treaty depth also promotes efficiency gains from coordinated defense effort. 
I now describe the role of treaty depth in more detail. 



\subsection{Alliance Treaty Depth} 


% Define depth again 
Alliance depth is the extent of defense cooperation formalized in the treaty. 
Deep alliances require additional policy coordination and military cooperation beyond a promise of military support. 
While shallow alliances stipulate more arms-length cooperation between members, deep treaties lead to closer cooperation. 
Defense cooperation in a deep alliance takes many forms. 
Allies can form an integrated military command, provide military aid, commit to a common defense policy, provide basing rights, set up an international organization or undertake companion military agreements. 
All of these obligations move alliance members away from an arms-length partnership towards close cooperation via policy coordination and regular interaction, while imposing monetary and policy autonomy costs.\footnote{Although depth can have monetary costs, reductions in military spending due to greater treaty credibility will often outweigh those costs. Such an effect is plausible, as contributions to alliances are a small relative to full defense budgets, and non-major powers rarely use basing rights in deep alliances to deploy their troops abroad.} 

 
One example of a deep alliance is a 1948 defense pact between the United Kingdom and Jordan, which includes unconditional military support, basing rights, military aid, official military contact, and an Anglo-Transjordan Joint Defense Board.  
This is a deeper alliance than a 1912 treaty between Greece and Bulgaria which only commits to mutual defense and consultation if either state is attacked by Turkey. 
Increasing military coordination adds ties between alliance members beyond a promise of military support.


% Emphasis: credibility
Treaty depth reduces non-major power military spending in two ways. 
First, depth reassures partners and reduces leverage on defense spending.  
Deep alliances are more credible because defense cooperation is costly. 
Making costly commitments of bases, policy coordination, or aid reassures allies. 
Depth is especially useful because alliance members face a time inconsistency problem. 
Alliance treaty fulfillment depends on shared foreign policy interests \citep{Morrow2000, Leeds2003a}, so changing foreign policy interests threaten alliance fulfillment \citep{LeedsSavun2007}. 
A deep alliance makes a series of repeated transfers, and states can signal commitment by maintaining those transfers.\footnote{Conversely, eliminating or reducing planned transfers reduces the credibility of the whole alliance.} 


% Efficiency argument
Credibility in deep alliances may also facilitate more efficient defense spending. 
States often use alliances to formulate joint war plans \citep{Poast2019a}, which allows alliance members to provide specific capabilities. 
Specialization means members of deep alliances spend less on the military but retain adequate security.
Alliance credibility and efficiency gains are inseparable. 
States will only specialize if they believe the alliance is reliable \citep{Leeds2003a}.  


On the other hand, shallow alliances are less credible, so participation in these treaties is more likely to increase military spending. 
These treaties have some basic credibility from hands-tying signals \citep{Fearon1997}, as well as the audience \citep{Morrow2000} and reputational \citep{Gibler2008, Crescenzietal2012} costs of violation.
Even so, threats to abandon low-spending allies are more credible than in a deep alliance where partners have taken pains to reassure their partners.  
In a shallow alliance, members must hedge against abandonment, which partners can use as leverage to discourage low defense spending. 
Maintaining the benefits of alliance participation then requires increased defense spending, because low military spending might endanger the treaty or expose states to adverse consequences if they are abandoned. 
Shallow alliances are also less likely to see coordination and specialization due to the fear of abandonment. 
As a result, participation in shallow alliances often increases military spending.\footnote{
One objection to this argument is that deep alliances are more valuable to members, which augments allies influence on defense spending. 
Although alliance value adds some leverage, it cannot offset reducing the credibility of threats to abandon low-spending allies.
Value increases leverage because states fear their allies will abrogate a valuable alliance, and deep alliances counteract this essential concern. 
}


% Cases
To illustrate the logic, consider two related alliances from the inter-war period. 
A 1920 treaty between France and Belgium (ATOPID 2055) added commitments of military aid and policy coordination to defensive obligations. 
Given this depth, the Franco-Belgian alliance reduced Belgian defense expenditures, even while they participated in occupying the Ruhr. 
A more limited treaty with only military support between France, Belgium, the United Kingdom, Italy and Germany (ATOPID 2130) increased Belgian spending, on the other hand.   
 
 
Taken together, the examples and the argument suggest that treaty depth modifies the impact of alliance participation on non-major power military spending. 
Shallow alliances often increase military spending, and deep alliances usually reduce spending.  
If we think about depth as a continuous variable, there should be a negative correlation between treaty depth and the impact of alliance participation on non-major power military spending as the positive effects of shallow treaties turn towards negative effects in the deepest alliances. 
This implies three separate hypotheses, one about shallow alliances, another about deep alliances, and the third about how changes in treaty depth modify the association between alliance participation and military spending.\footnote{Hypothesis 3 follows from Hypotheses 1 and 2.}
 

\begin{quote}
\textsc{Hypothesis 1: On average, participation in shallow alliances will increase percentage changes in non-major power military spending.}
\end{quote}

\begin{quote}
\textsc{Hypothesis 2: On average, participation in deep alliances will decrease percentage changes in non-major power military spending.}
\end{quote}

\begin{quote}
\textsc{Hypothesis 3: As alliance treaty depth increases, the impact of alliance participation on percentage changes in non-major power military spending will decrease.}
\end{quote}


Before detailing how I test these hypotheses, I must briefly justify two measurement decisions. 
First, the three hypotheses predict how percentage changes in non-major power military spending differ under deep and shallow alliances. 
Percentage changes in military spending express changes in spending as a share of the previous year's defense budget.
This variable is an appropriate outcome of interest, in part because it expresses the opportunity costs of military spending. 
All else equal,\footnote{Especially holding economic growth constant.} a larger increase in spending relative to the previous year's defense budget imposes more constraints on other goods. 
Using percentage changes also facilitates comparisons across diverse states and years. 


% Capabilitiy is alliance participation 
Second, I use allied capability to express the consequences of alliance participation for military spending, rather than a simple dichotomous indicator of participation.
This measurement strategy matches my argument and encompasses the division in research designs across previous studies. 
My argument starts with the premise that states form alliances so that allied capability supports their foreign policy goals. 
States do not respond to a treaty per se, rather they respond to expectations that allies will employ military capability on their behalf.
This makes alliances with more capable states are more valuable, all else equal.
Therefore, alliance participation affects military spending through joining an alliance and changing allied capability after the treaty forms. 
In previous scholarship, general research designs address the implications of joining an alliance, while specific designs focus on the ramifications of changes in allied capability. 
Conceptualizing alliance participation in terms of allied capability encapsulates both designs, creating a unified approach to understanding how alliances affect military spending. 


% Transition para
Because my argument focuses on differences between deep and shallow treaties, the research design must measure alliance treaty depth and show how depth modifies the impact of allied capability on military spending.  
I use a measurement model to infer treaty depth from formal content, then connect alliance characteristics to military spending with a multilevel model. 
The next section describes the research design in more detail. 



\section{Research Design} 


% two contributions: Develop latent depth measure and then put it into an ML model
The research design involves two steps. 
First, I develop a latent measure of treaty depth for alliances with military support. 
Second, I employ that measure in a multilevel model to estimate how treaty depth modifies the impact of alliance participation on military spending. 
I estimate the multilevel model in a sample of non-major powers from 1816 to 2007. 
The next section describes the measure of alliance treaty depth. 


\subsection{Measuring Alliance Treaty Depth} 


% Intuituion behind latent measures: observed char reflects underlying concept
I start by assuming that formal treaty commitments reflect alliance depth.\footnote{Formal treaty obligations may not be fully implemented, but formal depth is more likely to produce practical depth as states try to uphold the credibility of their commitments.}
I then use observed alliance treaty characteristics to infer depth, which could produce ordinal or continuous measures of treaty depth.
I now discuss a related ordinal measure and a related continuous measure of similar concepts. 


In an ordinal index of treaty depth, researchers theoretically assign different depth values. 
\citep{LeedsAnac2005} develop such an ordinal measure by assigning alliances military institutionalization scores of zero, one or two based on the extent of investment in joint action required by the alliance treaty. 
The resulting measure roughly captures treaty depth, but it understates variation in treaty depth. 
This measurement strategy imposes equal weights on different depth sources and does not aggregate multiple sources of depth. 
For example, it treats an integrated military command and military bases as equivalent sources of depth, and does not add the two sources together if both are present. 
I assess these theoretical assumptions with a more flexible measurement strategy.
 
 
I employ latent variable modeling to create a continuous measure of treaty depth that makes more nuanced distinctions between alliances.
The specific measurement model uses correlations between observable alliance treaty content and unobserved latent depth to predict the depth of each treaty. 
With this approach, theory identifies the relevant correlates of treaty depth, but the data drives how much depth each variable adds to the alliance. 


% Justify use 
Measurement models have a rich history in political science \citep{Clintonetal2004, TreierJackman2008, Fariss2014}.
My particular measure builds on work by \citet{BensonClinton2016}, who use a latent variable model \citep{Quinn2004} to measure alliance scope, depth and capability.  
I emulate Benson and Clinton's approach, but use a different concept, sample of alliances and estimator. 
Conceptually, \citet{BensonClinton2016} define depth as the costliness of the alliance in general, so they include measures of economic issue linkages and secrecy.
My definition of depth only includes military cooperation, because I view issue linkages as a separate source of credibility. 
Given their broad definition of depth, Benson and Clinton also include neutrality pacts in their sample of alliances.
I am only interested in offensive and defensive alliances, however.  
As for the estimator, in some latent variable models, the latent variables influence the form of the dependence structure and the marginal distributions of the latent value estimates. 
This can produce misleading inferences about latent scores, so I employ a different estimator that does not have this limitation \citep{Murrayetal2013}.


Due to the limits of ordinal measures and key conceptual differences with Benson and Clinton's latent measure, existing measures of treaty depth do not fit my purposes in this paper. 
See the appendix for a more detailed justification of this choice and evidence that \citet{LeedsAnac2005}'s measure of institutionalization produces similar inferences. 
I create a new measure of treaty depth in offensive and defensive ATOP alliances using a semiparametric factor analysis. 


% How the model works
I use a Bayesian Gaussian Copula Factor Model \citep{Murrayetal2013} to measure alliance treaty depth. 
Murray et al's model improves inferences from mixed factor analysis for continuous, ordinal, and binary observed data by relaxing distributional assumptions.
Given discrete observed variables and non-Gaussian latent variables, the dependence among the latent variables and their marginal distributions are both influenced by the latent variables.
This approach breaks the dependence between the latent factors and marginal distributions by using copulas to encode the dependence among the latent variables.\footnote{Copulas are a distribution function on $[0, 1]^p$ where each univariate marginal distribution is uniform on [0, 1].}
Beyond the semiparametric aspect, this measurement model is a standard ordinal factor analysis.


I estimated the measurement model using observed data from 289 alliances with offensive or defensive obligations in the alliance-level ATOP data \citep{Leedsetal2002}. 
I examine alliances with military support because prior studies of alliance participation and military spending focus on these treaties.
Indicators of treaty depth include military aid, bases, international organization formation, integrated military command, defense policy coordination, subordination of forces in wartime, specific contribution requirements, and commitments to form companion military agreements.\footnote{These are the variables \citet{LeedsAnac2005} use, with the addition of a companion military agreements dummy.}
The argument suggests there is a single factor underlying variation in all eight indicators, so I fit the model with one latent factor. 
To estimate the model, I used Parameter expanded Gibbs sampling, the default generalized double Pareto (GDP) prior, 20,000 burn-in iterations of the MCMC chain, and 30,000 samples thinned every 30 observations to ensure convergence. 
The estimates include posterior distributions for the factor loadings and the latent factor. 


% Show the measure
The latent variable model produces a useful measure of treaty depth. 
To summarize treaty depth, I use the posterior mean of the latent factor for each alliance, so each alliance has its own depth value.
The posterior mean captures the expected depth of an alliance treaty, conditional on its formal promises. 
\autoref{fig:ld-summary} describes the latent depth of ATOP alliances with defensive or offensive commitments from 1815 to 2016.
There is substantial variation in alliance treaty depth, which has several sources. 
The top panel in \autoref{fig:ld-summary} shows the factor loadings from the latent variable model, which are essentially the correlations between these observed variables and the latent factor. 
Policy coordination, integrated military command, and formal organizations are the three largest correlates of depth. 
The other five factors have roughly comparable associations with latent treaty depth. 


The bottom panel of \autoref{fig:ld-summary} plots the posterior means and uncertainty of the depth estimates against the start year of the treaty. 
Many treaties have no depth, and are clustered on around -0.8.  
171 alliances have a depth score higher than -0.6 because at least one source of depth is present. 
Even after accounting for uncertainty, it is possible to distinguish between some alliances. 


\begin{figure}
	\centering
		\includegraphics[width=0.95\textwidth]{../figures/ld-summary.png}
	\caption{Summary of the latent measure of alliance treaty depth for 289 defensive or offensive alliances from 1816 to 2016. The top panel plots the factor loadings with 90\% credible intervals. The bottom panel plots mean treaty depth (points) and the standard deviation (error bars) against the start year of the treaty.}
	\label{fig:ld-summary}
\end{figure}


% Cases- especially deep and shallow treaties
Although the values of the latent measure are not intrinsically meaningful, differences between treaties on the latent scale are informative. 
The median of treaty depth is -0.09, and the mean is 0.05. 
The median treaty is the Southeast Asian Treaty Organization (SEATO), which includes a formal international organization (ATOP ID 3260). 
There are many shallow treaties that only include military support. 
One such alliance is an 1855 pact between France, the UK and Sweden (ATOPID 1190) which promises defense and consultation. 


Three of the deepest treaties are a 1993 alliance between Russia and Tajikistan (ATOPID 4470), a 1958 alliance between the UAE and Yemen (ATOPID 3345), and a 1981 pact between Gambia and Senegal (ATOPID 3930). 
All these alliances stipulate extensive defense cooperation. 
The alliance between Russia and Tajikistan includes military aid, bases, a companion military agreement, and integrated military command. 
The other two treaties attempted to establish a federation through military support, international organizations, basing, and defense policy coordination. 


The distribution of treaty depth and the example alliances suggest that the latent measure has some face, concept, and discriminant validity. 
As an example of face validity, the Gambia-Senegal federation requires deeper cooperation than arms-length commitments of military support. 
Shallow treaties promise little beyond military support, matching my conceptualization of treaty depth. 
Last, \autoref{fig:ld-summary} shows that this measure can distinguish between deep and shallow commitments. 


The above measure of treaty depth is the key explanatory variable in my empirical analysis. 
My argument claims that differences in depth among alliances modify the impact of alliance participation on percentage changes in military spending by alliance members at the state-year level of analysis. 
The next section summarizes this estimation strategy. 


\subsection{Estimation: Multilevel Model} 


% Best fit for theoretical process. Can compare alliances. 
Because alliances and state-year changes in military spending are separate levels of analysis, I use a multilevel model to estimate the association between treaty depth and military spending.  
Multilevel modeling bridges levels of analysis \citep{SteenbergenJones2002, GelmanHill2007}. 
My model estimates heterogeneous effects of alliance participation on military spending as a function of alliance characteristics, in order to make inferences about how alliance characteristics like formal depth modify the impact of individual alliances on military spending. 
To facilitate computation and interpretation, I fit the model using Bayesian estimation in STAN \citep{Carpenteretal2016}. 
See the appendix for details of the weakly informative prior distributions and evidence the chains converged.


This research design is more complicated than a panel data model like the estimator \citet{DigiuseppePoast2016} use,\footnote{See the appendix for results from several models with state-level indicators of alliance depth.} but the multilevel components add substantial value, especially by connecting the argument and research design.
I argue that treaty depth modifies the impact of alliance participation on growth in military spending. 
Differences in how alliance participation impacts military spending are the outcome of interest.  
The multilevel model explicitly compares the impact of participation in deep and shallow alliances by estimating how the changes in treaty depth modify the consequences of alliance participation. 


Although the multilevel model matches the argument, standard panel models employ state-level proxies for alliance characteristics, which compare states rather than alliances.
This practice of aggregating alliances at the state-year level of analysis may produce misleading inferences \citep[pg. 356]{McElreath2016}.
Summarizing alliance characteristics at a different level of analysis changes the mean and variance of key independent variables, which could affect inferences. 
Multilevel modeling avoids this aggregation problem by retaining the structure of alliance data. 
One, states can participate in more than one alliance and alliances have heterogeneous effects on military spending.
The multilevel model estimates the specific impact of each alliance on members' military expenditures, which reveals differences between individual treaties. 
Aggregating multiple alliances into state level indicators will mask any heterogeneous effects of individual treaties.\footnote{Partial pooling of the alliance-specific parameters generates reasonable estimates for each alliance.} 


Furthermore, multiple alliance characteristics modify the consequences of alliance participation.
The multilevel model captures multiple sources of heterogeneity in how alliances impact military spending. 
In a panel estimator with state-level proxies for alliance characteristics, accounting for correlated aspects of alliance design is difficult. 
Treaty depth is correlated with other aspects of alliance membership and design, so this step is important.\footnote{For example, I show in another paper that democratic alliance membership is positively correlated with treaty depth.}
Panel estimates that account for one or two alliance characteristics can only do this by averaging different parts of a state's alliance portfolio or subsetting the data.
Averaging reduces theoretically interesting alliance-level variation, and analysis of multiple subsets risks generating spurious findings through multiple comparisons.  
In a multilevel model, I can account for how multiple alliance characteristics change the consequences of alliance participation by including other variables besides treaty depth in an alliance level regression. 
Therefore, my estimate of how treaty depth modifies the impact of alliance participation on military spending holds many key alliance and state characteristics constant. 
I now describe the model specification in more detail. 
 


\subsubsection{Model Specification} 

% Two separate but connected regressions
% State-level regression- alliances enter through spending matrix.
This multilevel model connects two distinct regressions. 
The base is a state-year-level regression, which includes the impact of alliance participation.
A second alliance-level regression modifies the effect of alliance participation on military spending, like an interaction. 


The state-year-level regression starts with a distribution for the outcome:
\begin{equation}
y \sim student_t(\nu, \mu, \sigma)
\end{equation}
 

$y$ is the dependent variable--- percentage changes in military spending. 
I model the outcome using a t-distribution with degrees of freedom $\nu$ to address heavy tails.\footnote{I estimate $\nu$ directly.}
$\sigma$ is analogous to the error term in a frequentist regression as it captures unexplained variation.  
$\mu$, the mean of the outcome, depends on several factors.
\begin{equation}
\mu = \alpha + \alpha^{st} + \alpha^{yr} +\textbf{W}_{n \times k} \gamma_{k \times 1}  + \textbf{Z}_{n \times a} \lambda_{a \times 1} 
\end{equation}


Percentage changes in spending are a function of an overall intercept $\alpha$, state and year varying intercepts $\alpha^{st}$ and $\alpha^{yr}$ and a matrix of state-level control variables $\textbf{W}$.
The $\textbf{Z} \lambda$ term incorporates alliance participation.
$\textbf{Z}$ is a matrix of state participation in alliances. 
Columns correspond to each of the $a$ alliances in the data, and rows to state-year observations. 
For this sample of non-major powers, $\textbf{Z}$ has 190 columns and 8,290 rows. 
If a state is not in the alliance, the corresponding cell of the matrix is zero.
If a state is part of the alliance in a given year, the matrix element contains the log of total allied military spending, which is normalized by year.\footnote{Normalization keeps the parameters on similar scales, which is important for modeling. I selected normalization by the annual maximum, theoretically and corroborated this choice by comparing models fit with different ways of expressing allied capability. See the appendix for details.} 
I use total allied spending to express the effect of alliance participation to match the theoretical emphasis on allied capability. 
$\textbf{Z}$ encodes a quasi-spatial indicator of alliance participation for all $a$ alliances in the data. 
States can be members of multiple treaties at once, so observations are not neatly nested. 
This specification means each alliance has a unique impact on military spending, even when states participate in multiple treaties. 


$\lambda$ is a vector of parameters which estimate the impact of participation in specific alliances on military spending. 
Because the non-zero elements of $\textbf{Z}$ are allied spending, the $\lambda$ parameters capture alliance members' response to allied capability. 
Each alliance has a unique $\lambda$, so there are 190 $\lambda$ estimates. 
The $\lambda$ parameters have shared distribution, so I assume alliances are similar but different in how they impact military spending. 


% Alliance-level regression:
The second part of the multilevel model uses alliance characteristics to predict how alliance participation is associated with percentage changes in military spending. 
The $\lambda$ parameters are the outcome in an alliance-level regression.
As a result, the impact of alliance participation on members' military spending depends on treaty characteristics, including depth. 
In this second-level regression: 


\begin{equation}
\lambda_{a} \sim N(\theta_{a}, \sigma_{all})
\end{equation} 
and 
\begin{equation}
\theta_{a} = \alpha_{all} + \beta_1 \mbox{treaty depth} + \textbf{X}_{a \times l} \beta
\end{equation}


% Like an interaction between alliance and state-level factors 
In the alliance-level regression, $\textbf{X}$ is a matrix of the $l$ alliance-level control variables and $\alpha_{all}$ is the constant.
Adding $\sigma_{all}$ means predictions of $\lambda$ are not deterministic--- the alliance level regression contains an error term. 
A larger $\sigma_{all}$ indicates more variation in how alliance participation impacts military spending. 


The second-level regression includes treaty depth, and each $\beta$ parameter modifies the impact of alliance participation on percentage changes in military spending. 
The $\beta$s are like marginal effects in an interaction. 
Treaty depth impacts military spending by modifying the consequences of alliance participation. 
Changing treaty depth shifts $\lambda$, which in turn affects military spending.
Hypothesis 3 predicts $\beta_1$ will be negative for non-major powers. 


In this model, the $\lambda$ parameters express heterogeneous effects of participation in individual alliances.
The $\beta$ parameters capture how alliance characteristics modify the impact of alliance participation on military spending.  
Again, using alliance characteristics to predict the impact of alliance participation matches my conditional argument. 
I now describe the sample and key variables in the analysis.  



\subsection{Sample and Key Variables} 

% Sample of states & alliances: restricted to treaties with military support
I estimate the multilevel model on a sample of non-major power states from 1816 to 2007. 
I identify non-major powers using a measure of major power status from the Correlates of War Project. 
Alliance participation data comes from the ATOP project \citep{Leedsetal2002}.  
I focus on participation in defensive and offensive treaties, because prior studies of alliances and military spending examine these treaties. 
The sample contains data from 8,280 state-year observations and 190 alliances. 
Therefore, $\textbf{Z}$ is has 8,280 rows and 190 columns. 


% DV: percentage changes in milex
The dependent variable is percent changes in military spending, which is calculated as:
\begin{equation}
\mbox{\% Change Mil. Expend} = \frac{ \mbox{Change Mil. Expend}_t }{ \mbox{Mil. Expend}_{t-1} }
\end{equation} 
I used the Correlates of War Project's data on military spending to measure percentage changes in spending \citep{SingerCINC1988}.\footnote{Estimating the model on SIPRI military spending data produces similar results: see the appendix for details.} 
The annual percentage change in spending equals that year's change in spending as a share of the previous year's military spending.
Thus, annual changes are bench marked to previous spending levels. 
To facilitate model fitting, I apply the inverse hyperbolic sine transformation to this variable.\footnote{This transformation applies to positive, negative and zero values. It has minimal impact on values between -1 and 1, but pulls in large positive values, which range as high as 140. Inferences about treaty depth and other alliance characteristics are comparable with and without the transformation.}
Using percentage changes in military expenditures as the dependent variable helps the research design. 
The level of military spending is not stationary for most states, especially in longer panels. 
Thus, using percentage changes in spending reduces the risk of spurious inferences.
Benchmarking changes to prior expenditures also facilitates comparisons across states and over time. 


% key IV: mean treaty depth
The key independent variable is the mean latent depth of each alliance, based on the measurement model. 
This variable enters the model in the alliance-level regression and I expect it will have a negative coefficient. 
I also include several state and alliance-level controls.


In the state-level regression, I adjust for several correlates of alliance participation and military spending. 
State-level covariates include GDP growth \citep{Boltetal2018} regime type, international war \citep{Reiteretal2016}, civil war participation \citep{SarkeesWayman2010}, annual MIDs \citep{Gibleretal2016}, rival military spending \citep{ThompsonDreyer2012} and a dummy for Cold War years.
Conflict participation, alliances, and military spending are all correlated \citep{SeneseVasquez2008}.
I include growth in GDP instead of levels because GDP levels are non-stationary and economic growth shapes the opportunity costs of military spending \citep{Kimball2010, Zielinskietal2017}.  

 
Other alliance level variables are correlates of treaty design and military spending, including the number of members and share of democracies in a treaty at time of formation \citep{Chibaetal2015}. 
I control for issue linkages by creating a dummy indicator of whether the alliance promises any kind of economic cooperation \citep{Poast2013, LongLeeds2006}. 
As an indicator of hierarchical security relationships, I include a count of foreign policy concessions in the alliance. 
I also mark the presence of unconditional military support using a dummy variable I constructed using existing indicators of conditional support in the ATOP data. 
Because threat may drive states to form deeper alliances and affect subsequent military spending, I control for the average threat of alliance members at the time of alliance formation using the threat measure of \citet{LeedsSavun2007}. 
I adjust for superpower membership--- whether the United States or Soviet Union participated in a treaty during the Cold War. 
Two dummy indicators of wartime alliances and asymmetric obligations \citep{Leedsetal2002} complete the alliance-level regression specification. 
Though I discuss these variables as controls, many of them are theoretically interesting in their own right. 
Having described the measure of treaty depth, multilevel model, and covariates, I now turn to the results of the analysis. 

 

\section{Results}


This section summarizes inferences from the multilevel model. 
I find support for all three hypotheses. 
Because shallow alliances tend to increase military spending and deep alliances often decrease spending, treaty depth and the effect of alliance participation on non-major power military spending are negatively correlated. 
Results are based on 2,000 samples from four chains, with 1,000 warm-up iterations. 
To facilitate model fitting, I employed a non-centered parameterization of the varying intercepts and a sparse matrix representation of \textbf{Z}. 
Standard convergence diagnostics indicate the chains adequately explored the posterior.\footnote{See the appendix for details on convergence and other robustness checks.} 


% note on interpreting Bayesian results
Because I used Bayesian modeling to estimate the association between treaty depth and percentage changes in military spending, each coefficient has a posterior distribution--- the likely values of the coefficient conditional on the priors and observed data.
There are no indicators of statistical significance. 
Instead, I use the 90\% credible intervals of the parameters and calculate the negative posterior probability for the treaty depth coefficient to assess Hypothesis 3.\footnote{I use 90\% intervals because inferences about 95\% intervals are sensitive to simulation variance in Bayesian analysis.}


\autoref{fig:results-allreg} summarizes the coefficient estimates from the alliance-level regression and a simulated substantive effect of a large increase in treaty depth. 
The preponderance of evidence matches Hypothesis 3, as shown in the top panel of \autoref{fig:results-allreg}.
There is a 96\% chance treaty depth is negatively correlated with the impact of alliance participation on percent changes in military spending for non-major powers.
Also, the 90\% credible interval for treaty depth does not include zero. 


\begin{figure}[htbp]
	\centering
		\includegraphics[width=0.95\textwidth]{../figures/results-allreg.png}
	\caption{Summary of alliance-level regression results from the multilevel model of alliance participation and military spending. The top panel shows the 90\% credible intervals to summarize the posterior densities of coefficients in the alliance-level regression. The bottom panel plots the estimated effect of participation in an alliance with average capability on growth in military spending for a deep and shallow treaty, as well as the difference between deep and shallow alliances. In both panels, points mark the posterior mean, and the bars encapsulate the width of the 90\% credible interval.}
	\label{fig:results-allreg}
\end{figure}


This alliance-level depth coefficient has a substantively important effect on growth in military spending, which I summarize in the bottom panel of \autoref{fig:results-allreg}. 
I assess this substantive effect by simulating the effect of changing treaty depth from the minimum value of -0.8 to 1.5, which is in the fourth quartile. 
Holding other alliance covariates at their modes or medians, this increase in depth reduces a hypothetical alliance parameter $\lambda$ by .08 in expectation, which then alters military spending growth.
The results of the the simulation are summarized with 90\% intervals for predicted growth in military spending in the hypothetical shallow alliance, predicted spending growth in the hypothetical deep alliance, and the difference between those two scenarios.
Assuming the hypothetical alliance has median capability, the difference in spending growth between the shallow and deep treaty has a mean of -.03.
The 90\% credible interval of this predicted fall in military spending due to increasing treaty depth ranges from -0.056 to -0.001.
There is a perceptible difference between non-major power military spending growth in deep and shallow alliances, all else equal.  


To assess Hypotheses 1 and 2, I examine patterns in the $\lambda$ parameters across the range of treaty depth.
Each $\lambda$ measures the impact of treaty participation, so if treaty depth has a large influence on alliance participation, it will appear in the $\lambda$ estimates. 
On average, participation in deep alliances should have a negative effect on members' percent changes in military spending and shallow alliances should have a positive effect.
Therefore, there should be a negative trend in the expected value of $\lambda$ as treaty depth increases.


\begin{figure}[htbp]
	\centering
		\includegraphics[width=0.95\textwidth]{../figures/results-allpred.png}
	\caption{Summary of the predicted effect of alliance participation across the observed values of treaty depth for 192 alliances from 1816 to 2007. The top panel plots the mean of each $\lambda$ parameter by treaty depth. The bottom panel plots 9,128 state-alliance-year predictions of growth in military spending from alliance participation, which combines the $\lambda$ value for the alliance and allied capability value for that state-year observation. Darker colors indicate more data points in the particular hexagon.}
	\label{fig:results-allpred}
\end{figure}


There preponderance of evidence in \autoref{fig:results-allpred} matches the predictions of Hypotheses 1 and 2, which plots the $lambda$ estimates and predicted impact of alliance participation on military spending growth. 
The top panel of \autoref{fig:results-allpred} plots the expected value of $\lambda$ across the range of treaty depth. 
As expected, shallow treaties often have positive $\lambda$ values for non-major powers,\footnote{All the negative $\lambda$ estimates in alliances with minimal depth are treaties between the Soviet Union and Eastern European states during the Cold War.} which corresponds to Hypothesis 1. 
Most of the deepest treaties have a negative $\lambda$, which matches Hypothesis 2. 
Because other treaty characteristics and chance also influence the $\lambda$ estimates, there is tremendous variation in how alliance participation impacts non-major power military spending. 


I then used the $\lambda$ estimates to predict the impact of alliance participation on state-year military spending growth. 
Because they express the impact of allied capability, the $\lambda$ values are not direct predictions of how alliance participation affects military spending. 
To assess the predicted annual impact of different alliances, I multiplied the alliance membership matrix $\textbf{Z}$ by the $\lambda$ parameters to generate 9,124 state-alliance-year predictions.\footnote{These estimates hold all state-level covariates like threat and regime type constant.} 
These predictions show how each $\lambda$ translates into military spending growth. 



The scatter plot in the bottom panel of \autoref{fig:results-allpred} shows the distribution of predicted changes in military spending from alliance participation.
Each point marks the mean estimated effect of participation in an alliance on growth in military spending for that year.
To avoid overplotting the 9,124 estimates, I combined the points into hexagons. 
Darker hexagons mark areas with more points. 
As Hypothesis 1 predicts, participation in shallow alliances regularly increases military spending. 
Many state year alliances with shallow alliances see little effect on military spending, however. 
As treaty depth increases, more alliances reduce military spending growth among their members.\footnote{Wartime alliances are the main exception to this trend.}
Some of the deepest alliances have a negligible effect on military spending despite highly negative $\lambda$ values because they have comparatively little capability. 
The same pattern holds in the top and bottom panels of \autoref{fig:results-allpred}, which corresponds to the expectations of Hypotheses 1 and 2. 


In summary, I find that treaty depth modifies the impact of alliance participation on military spending.  
Participating in deep alliances often reduces military spending, while being part of a shallow alliance often increases it. 
This has important consequences for our view of alliance participation and military spending. 
I now discuss the implications of the argument and results. 



\section{Discussion and Conclusion}


% Precise interpretation: compares alliances. Not treaty vs absence. 
My findings add to our understanding of alliance participation and military spending and address debates over whether alliance participation increases or decreases military spending. 
Claims alliance participation only increases or decreases military spending are incomplete. 
My argument shows how treaty depth modifies the impact of alliance participation on military spending, which builds on other conditional arguments \citep{DigiuseppePoast2016}. 
I show that whether alliance participation increases or decreases military spending depends on treaty depth. 
Compared to no alliance at all, joining a shallow treaty usually increases military expenditures, while participation in a deep alliance often lowers defense spending. 


% two results that are new/unanticipated 
There are two other noteworthy findings.  
First, there are many alliances that increase non-major power military spending, which cuts against popular expectations that non-major powers use alliances to reduce defense spending. 
To gain security from alliance participation, non-major powers may need to increase their defense budget \citep{Horowitzetal2017}. 
Second, the military spending growth predictions in \autoref{fig:results-allpred} suggest that many alliances have little effect on military spending, which existing theories and my own argument have some difficulty explaining. 
These negligible effects may reflect competing effects from different parts of alliance treaty design and membership, or general concerns about the credibility of alliance treaties. 


Of course, these findings are derived from a novel research design. 
Given my departure from previous research designs, how should we compare these results to prior evidence on alliance participation and military spending? 
Connecting my results with earlier evidence requires renewed attention to specific and general research designs. 
Recall that general studies compare states in an alliance to those without one in a global sample and specific studies estimate responses to allied military spending in a few treaties. 
The results encompass specific and general research designs by using allied capability to measure alliance participation. 
Each impact of alliance participation includes the effect of joining an alliance and changes in allied capability. 
I then use the alliance-level regression to understand how the impact of alliance participation varies with treaty design and membership.   


Although my research design advances the debate on alliance participation and military spending, it has two limitations. 
First, my findings only address formal treaty depth. 
The measure of treaty depth only includes formal promises, in part because informal depth is harder to observe. 
As a result, my test of alliance depth may be conservative--- it does not capture phenomena that should have a similar effect. 
It may be also overstate the findings if formal depth is not implemented, however. 
Strategic alliance design is the second possible weakness of the test. 
Domestic politics can affect alliance obligations, for example \citep{Davis2004, Chibaetal2015}.   
To address this issue, I controlled for correlates of alliance participation and treaty depth at each level of the model, with a particular focus on factors like democracy, alliance size, external threat, and other sources of credibility.
At the state level, I adjusted for threat, economic growth, and regime type, all of which are possible correlates of treaty depth and growth in military spending. 
Even with this effort, selection into different alliances could still produce unobserved differences between alliances I cannot adjust for. 


Despite these limitations, the argument and results generate valuable insights about alliance participation and military spending. 
I explain when alliance participation is associated with increases or decreases in military spending among non-major powers, which addresses a debate between contradictory views of alliances.  
I provide evidence that how alliance participation impacts military spending depends on state capability and alliance treaty depth using a new measure of alliance treaty depth and a multilevel model. 


% Concluding implications: scholarship 
My findings have implications for scholarship. 
First, my argument and findings reinforce the importance of accounting for heterogeneity among alliances.
Also, my research design could apply to other international institutions where institutional design shapes the consequences of participation.
Besides these general implications, the results raise interesting questions for future research. 
Why do some many alliances have a negligible effect on non-major power military spending? 
Because arms and alliances both provide security, existing arguments expect the two are connected, but I find little substantive effect in many alliances.
The domestic economic and political consequences of alliance participation remain relatively unexplored.
This could include how changes in allied military spending affect public support for alliance participation and the economic consequences of changes in military spending.  


% The argument indicates tradeoff: set up policy implication
Besides their scholarly value, the argument and evidence help inform policy debates about military spending. 
My argument claims that reassurance from deep alliances leads to lower defense spending. 
States can use deep cooperation to increase alliance credibility, but allies are less likely to increase military spending as a result. 
Therefore, there is tradeoff between treaty credibility and allied military spending. 


% Implications for policy. 
The United States is currently wrestling with the credibility-military spending tradeoff. 
Washington has often decried allies who provide too little for their own defense \citep{Lanoszka2015}. 
But allies are able to maintain low military spending partly because the United States makes deep commitments. 
Reducing the depth of US alliances could generate credibility problems, however. 
Low allied defense spending may be the price of credible commitments.  
Therefore, this paper is not an unconditional call to reduce the depth of US alliances. 
Adjusting existing treaties may be more difficult than designing new alliances and will have other ramifications. 
The full consequences of shifting treaty depth require additional scrutiny. 

 



\singlespace
 
\bibliography{../../MasterBibliography} 





\end{document}
