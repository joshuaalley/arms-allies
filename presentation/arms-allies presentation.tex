\documentclass{beamer}


\usetheme[progressbar=frametitle]{metropolis}
\usepackage{appendixnumberbeamer}

\usepackage{booktabs}
\usepackage[scale=2]{ccicons}

%\usepackage{pgfplots}
%\usepgfplotslibrary{dateplot}

\usepackage{xspace}
\newcommand{\themename}{\textbf{\textsc{metropolis}}\xspace}

%\setbeamertemplate{footline} % To remove the footer line in all slides uncomment this line
%\setbeamertemplate{footline}[page number] % To replace the footer line in all slides with a simple slide count uncomment this line

%\setbeamertemplate{navigation symbols}{} % To remove the navigation symbols from the bottom of all slides uncomment this line


\usepackage{graphicx} % Allows including images
\usepackage{grffile}
\usepackage{amsmath}
\usepackage{adjustbox} 
% have to have Mozilla's=Fira Sans} font and XeTeX installed to use full typography.

%----------------------------------------------------------------------------------------
%	TITLE PAGE
%----------------------------------------------------------------------------------------

\title{Alliance Participation and Military Spending}
\date{\today}
\author{Joshua Alley}
\institute{Texas A\&M University}


\begin{document}

 \maketitle


%----------------------------------------------------------------------------------------
%	PRESENTATION SLIDES
%----------------------------------------------------------------------------------------


%------------------------------------------------
% Here's my point. 
 \begin{frame}[standout]

How alliance participation affects military spending depends on treaty scope and state capability. 

 \end{frame}

%------------------------------------------------
% The two subpoints 
 \begin{frame}[standout]

\setbeamercovered{invisible}

\uncover<1-2>{1: Though alliance participation usually increases major power military spending, growth is lower in broad treaties.} 

\uncover<2>{2: Though alliance participation usually decreases non-major power military spending, growth is higher in broad treaties.}

 \end{frame}

%------------------------------------------------

\begin{frame}{Why Should You Care?}

\begin{figure}[htbp]
	\centering
		\includegraphics[width=0.95\textwidth]{mattis-nato.jpg}
	\label{fig:mattis-nato}
\end{figure}


\end{frame}

%------------------------------------------------

\begin{frame}[standout]

\huge \textit{Does alliance participation increase military spending?} \uncover<2->{Or decrease it?}

 \end{frame}

%------------------------------------------------

\begin{frame}{Competing Results}

\begin{table}[hbt!]
\begin{center}
\begin{tabular}{lccc}
     & Decrease & Increase & Null \\
\hline
Most \& Siverson 1987  &  &  & X \\
Conybeare 1994 & X & &  \\
Diehl 1994 &  & X &  \\
Goldsmith 2003 &  &  & X \\
Morgan \& Palmer 2006 &  & X & \\ 
Quiroz-Flores 2011 &  & X &  \\ 
Digiuseppe \& Poast 2016 & X &  & \\ 
Horowitz et al 2017 &  & X & \\ 
\hline
\end{tabular}
\end{center} 
\end{table}


 \end{frame}

%------------------------------------------------

\begin{frame}{Omission: Alliance Heterogeneity}

\begin{figure}
	\centering
		\includegraphics[width=0.95\textwidth]{lambda-est-full.png}
	\label{fig:lambda-est-full}
\end{figure}


 \end{frame}

%------------------------------------------------
 \begin{frame}[standout]

I use treaty scope and state capability to explain some of these differences between alliances. 

 \end{frame}



%------------------------------------------------

\begin{frame}{Outline}

I make my claim about alliance participation and military spending in three ways: 

\pause
\begin{enumerate}
\item Argument: Treaty Scope and State Capability
\pause
\item Statistical Analysis
\pause
\item Illustrative Argument Using NATO
\end{enumerate}


\end{frame}

%------------------------------------------------

\section{Argument}

%------------------------------------------------

\begin{frame}{Assumptions}

\begin{itemize} 
\item States pursue domestic consumption and foreign policy goods. 
\pause 
\item Security and influence are the two main foreign policy goods, which states get through alliances and military spending.  
\pause 
\item Military spending has opportunity costs, which decrease with state size. 
\pause 
\item Alliances are a costly signal of shared foreign policy interests: credible commitment to intervene.  
\end{itemize}


\end{frame}

%------------------------------------------------

\begin{frame}{Treaty Scope}

Not all alliances are equally broad. Scope depends on: 

\begin{enumerate} 
\pause
\item Conditions on military support. 
\pause
\item Other costly promises of cooperation. 
\end{enumerate}  

\end{frame}


%------------------------------------------------

\begin{frame}{Implications of Treaty Scope}

Greater treaty scope generates a tradeoff between: 

\begin{enumerate} 
\pause
\item Foreign policy gains. 
\pause
\item Freedom of action. 
\end{enumerate}  

\end{frame}

%------------------------------------------------

\begin{frame}[standout]

The implications of treaty scope depend on state capability. 

\end{frame}


%------------------------------------------------

\begin{frame}{State Capability: Major Powers}

\begin{itemize}
\item Alliances \& Spending: External Influence
\pause
\item Influence from altering the expected or actual outcome of conflicts. 
\pause 
\item Alliance participation usually \emph{increases military spending} 
\end{itemize} 


\end{frame}

%------------------------------------------------

\begin{frame}{Treaty Scope and Major Powers}

\begin{itemize} 
\item Broad treaties $\uparrow$ influence without $\uparrow$ spending.
\pause
\item Influence from issue linkages. 
\end{itemize} 


\end{frame}



%------------------------------------------------

\begin{frame}[standout]


\begin{quote}
\textsc{Hypothesis 1}: As alliance treaty scope increases, growth in major power military spending from alliance participation will decrease. 
\end{quote}

\end{frame} 

%------------------------------------------------

\begin{frame}{State Capability: Non-Major Powers}

\begin{itemize}
\item Alliances \& Spending: Territorial Security.
\pause
\item Replace domestic expenditure with allied capability.
\pause
\item Alliance participation usually \emph{decreases} military spending. 
\end{itemize} 

\end{frame}

%------------------------------------------------

\begin{frame}{Treaty Scope and Non-Major Powers}

\begin{itemize}
\item Broad treaties restrict freedom of action.
\pause 
\item Alliance is more valuable.
\pause
\item Allies have more influence.
\end{itemize} 

\end{frame}

%------------------------------------------------

\begin{frame}[standout]

\begin{quote}
\textsc{Hypothesis 2}: As alliance treaty scope increases, growth in non-major power military spending from alliance participation will increase. 
\end{quote} 


\end{frame}

%------------------------------------------------

\section{Empirical Analysis} 

%-----------------------------------------------

\begin{frame}{Research Design}

I need two things to test these predictions: 

\pause 
\begin{enumerate} 
\item Measure of treaty scope--- measurement model. 
\pause
\item Connect alliance-level variation with state-level outcomes--- multilevel analysis.  
\end{enumerate} 


\end{frame}

%------------------------------------------------

\begin{frame}{Measuring Treaty Scope}

I use a latent variable model to infer treaty scope from observed promises. 

\pause 

The posterior mean of the latent factor measures scope for each alliance. 

\end{frame} 

%------------------------------------------------

\begin{frame}{Details of Measure}
 
\begin{itemize}
\item Multiple observed indicators of scope (ATOP): 
\begin{itemize} 
\item \textit{Military Support}: offense, defense, neutrality, consultation, non-aggression, unconditional military support.
\item \textit{Other Cooperation}: bases, integrated command, economic/military aid, IO formation, conclude multiple other agreements, no other alliances. 
\end{itemize} 
\pause 
\item Semiparametric mixed factor analysis. (Murray et al 2013)
\pause
\item Generates a posterior distribution of scope for each alliance.
\end{itemize} 


\end{frame} 

%------------------------------------------------

\begin{frame}{Latent Measure of Treaty Scope}

% Visual summary of latent measure
\begin{figure}[htbp]
	\centering
		\includegraphics[width=0.95\textwidth]{ls-hist.png}
\end{figure}


\end{frame} 

%------------------------------------------------

\begin{frame}{Latent Measure of Treaty Scope: Narrow}

% Visual summary of latent measure
\begin{figure}[htbp]
	\centering
		\includegraphics[width=0.95\textwidth]{ls-hist-narrow.png}
\end{figure}


\end{frame} 

%------------------------------------------------

\begin{frame}{Latent Measure of Treaty Scope: Typical}

% Visual summary of latent measure
\begin{figure}[htbp]
	\centering
		\includegraphics[width=0.95\textwidth]{ls-hist-median.png}
\end{figure}


\end{frame} 

%------------------------------------------------

\begin{frame}{Latent Measure of Treaty Scope: Broad}

% Visual summary of latent measure
\begin{figure}[htbp]
	\centering
		\includegraphics[width=0.95\textwidth]{ls-hist-broad.png}
\end{figure}


\end{frame} 


%------------------------------------------------

\begin{frame}{Empirical Analysis: Multilevel Model}

\begin{itemize} 
\item Link alliance-level variation with state-level outcomes. 
\pause
\item Two connected regressions: alliance and state-level. 
\pause 
\item Alliance characteristics modify the association between alliance membership and spending growth.  
\end{itemize} 

\end{frame} 


%------------------------------------------------

\begin{frame}{ML Model}

\[
\begin{array}{cccccc}
\uncover<2->{ & & & & &\mbox{Alliance} \\
& & & & &    \mbox{Characteristics}  \\
\uncover<3->{& & & & \lambda = & \alpha_{all} + \beta_1 \mbox{Scope} + \textbf{X} \beta \\}
& & & & &    \downarrow  \\}
\mbox{Growth} =     & \mbox{Varying}   & + & \mbox{State}   & + & \mbox{Alliance} \\
\mbox{Mil. Ex.}      & \mbox{Intercepts}&   &  \mbox{Vars.} &   & \mbox{Participation} \\
\uncover<3->{\mbox{y} = & \alpha + \alpha^{st} + \alpha^{yr}   & + & \textbf{W} \gamma  & + & \textbf{Z} \lambda \\}
\end{array}
\]


\end{frame}

%------------------------------------------------


\begin{frame}{ML Model Specification}

\begin{equation}
y \sim student_t(\nu, \mu, \sigma)
\end{equation}
\begin{equation}
\mu = \alpha + \alpha^{st} + \alpha^{yr} +\textbf{W}_{n \times k} \gamma + \textbf{Z}_{n \times a} \lambda
\end{equation}

\begin{equation}
\lambda_{a} \sim N(\theta_{a}, \sigma_{all})
\end{equation}
\begin{equation}
\theta = \alpha_{all} + \beta_1 \mbox{Treaty Scope} + \textbf{X}_{a \times l} \beta
\end{equation}


\end{frame}


%------------------------------------------------

\begin{frame}{Example}

\setbeamercovered{transparent}

\begin{equation*}
\uncover<2->{\mu_{it} =} \uncover<3>{\alpha} \uncover<4>{+ \alpha^{st} + \alpha^{yr}} \uncover<5>{+ W_{it} \gamma} \uncover<6>{+ Z_{it} \lambda}
\end{equation*}

Example year: 

\begin{equation*}
\begin{split}
& \uncover<2->{\mbox{Argentina 1955} = } \uncover<3>{\mbox{Overall mean}} \\
& \uncover<4>{+ \mbox{Argentine Intercept} + \mbox{1955 Intercept}} \\
& \uncover<5>{+ \mbox{Argentine Characteristics}} \\
& \uncover<6>{+ \lambda_{OAS} * \mbox{OAS Expenditure} + \lambda_{Rio} * \mbox{Rio Pact Expenditure}}
\end{split}
\end{equation*}

\uncover<7>{
\begin{equation*}
\lambda_{Rio} = \alpha_{all} + \beta_1 \mbox{Treaty Scope} + \mbox{Controls}
\end{equation*}
} 


\end{frame}


%------------------------------------------------


\begin{frame}[standout]{Z} 

\begin{tabular}{lccc}
State-Year & Rio Pact & Warsaw Pact & \ldots \\
\hline
Argentina 1954 & .347 & 0 & \ldots \\
Argentina 1955 & .418  & 0 & \ldots  \\
 \vdots & \vdots & \vdots & \ldots  
\end{tabular}

 \end{frame}



%------------------------------------------------

\begin{frame}{Sample and Key Variables}

\begin{itemize}
\item \textbf{Split Sample}: major and non-major power states--- 1816-2007. Alliances with military support. 
\pause
\item \textbf{DV}: Growth in Military Spending $ = \frac{ \mbox{Change Mil. Expend}_t }{ \mbox{Mil. Expend}_{t-1} }$ 
\pause
\item \textbf{Alliance-Level IV}: Mean Treaty Scope
\end{itemize} 

\end{frame}


%------------------------------------------------

\begin{frame}{Controls}

\begin{itemize}
\item \textbf{State-Level Controls}: Interstate war, Civil War, Annual MIDs, GDP growth, POLITY, Cold War, Rival military expenditures. 
\pause 
\item \textbf{Alliance-Level Controls}: Share of Democracies, Number of Members, wartime, asymmetric obligations, US member (Cold War), USSR member.

\end{itemize} 

\end{frame}


%------------------------------------------------

\subsection{Results}

%------------------------------------------------

\begin{frame}{Association Between Treaty Scope and Growth in Military Spending} 

\begin{figure}
	\centering
		\includegraphics[width=0.95\textwidth]{str-post.png}
	\label{fig:str-post}
\end{figure}


\end{frame}


%------------------------------------------------

\begin{frame}[standout]{Importance} 

\begin{tabular}{lcc}
Sample & Posterior Mean & Median Ex. Growth \\
\hline
Major & -0.05 & 0.04 \\
\pause
Non-major & 0.03 & 0.06  \\
\end{tabular}

\pause

US spent \$36.0 billion on NATO in 2018, or 5.5\% of the total defense spending. 


\end{frame}

%------------------------------------------------


\begin{frame}{Treaty Scope and $\lambda$: Major Powers}

\begin{figure}[htbp]
	\centering
		\includegraphics[width=0.95\textwidth]{ls-lambda-maj.png}
	\label{fig:ls-lambda-maj}
\end{figure}


\end{frame}


%------------------------------------------------

\begin{frame}{Treaty Scope and $\lambda$: Non-major Powers}

\begin{figure}
	\centering
		\includegraphics[width=0.95\textwidth]{ls-lambda-min.png}
	\label{fig:ls-lambda-min}
\end{figure}


\end{frame}

%------------------------------------------------

\section{NATO}

%------------------------------------------------

\begin{frame}{Foreign Entanglement and Formal Obligations}

\begin{figure}
	\centering
		\includegraphics[width=0.95\textwidth]{acheson-nato-sign.jpg}
	\label{fig:acheson-nato-sign}
\end{figure}


\end{frame}

%------------------------------------------------

\begin{frame}[standout]

\large ``The Parties agree that an armed attack against one or more of them in Europe or North America shall be considered an attack against them all...'' 

 \end{frame}

%------------------------------------------------

\begin{frame}[standout]

\large ``assist the Party or Parties so attacked by taking forthwith, individually and in concert with the other Parties, such action as it deems necessary, including the use of armed force'' 

 \end{frame}

%------------------------------------------------

\begin{frame}[standout]

\huge ``such action as it deems necessary, including the use of armed force'' 

 \end{frame}

%------------------------------------------------

\begin{frame}{NATO Scope} 

\begin{figure}
	\centering
		\includegraphics[width=0.95\textwidth]{ls-hist-nato.png}
\end{figure}


 \end{frame}

%------------------------------------------------

\begin{frame}{Impact of NATO on Growth in US Military Spending} 

\begin{figure}
	\centering
		\includegraphics[width=0.95\textwidth]{nato-imp-us.png}
\end{figure}

\end{frame}


%-----------------------------------------------

\section{Conclusion}

%-----------------------------------------------

\begin{frame}{Conclusion}

How alliance participation affects military spending depends on state capability and treaty scope.  

\end{frame}


%-----------------------------------------------

\begin{frame}{Implication: What to do with US alliances?}

\begin{figure}[htbp]
	\centering
		\includegraphics[width=0.95\textwidth]{nato-map.jpg}
\end{figure}


\end{frame}


%-----------------------------------------------

\begin{frame}{Alliance Participation and US Military Spending}

\begin{figure}[htbp]
	\centering
		\includegraphics[width=0.95\textwidth]{us-agg-imp.png}
\end{figure}

\end{frame}

%-----------------------------------------------

\section{Looking Ahead}

%-----------------------------------------------

\begin{frame}{Dissertation}

This paper is part of a more general theory of alliance participation and military spending. 

\end{frame}


%-----------------------------------------------


\begin{frame}{My Research Agenda}

The political economy of security, with a focus on formal institutions. 

\begin{columns}

% Major powers
\begin{column}{0.5\textwidth}
\textbf{International Security}
\begin{itemize} 
\item Alliance Participation and Military Spending 
\item Reassessing the Public Goods Theory of Alliances
\end{itemize} 
\end{column}



\begin{column}{0.5\textwidth}
\textbf{Intra-State Conflict}
\begin{itemize}
\item Conflict Management Institutions and FDI
\item Sanctioning Terrorist Groups: Can it Work?
\item Weapon of the Weak?: Rebel Groups' International Law Talk, 1974-2011
\end{itemize} 
\end{column}

\end{columns}
 

\end{frame}


%-----------------------------------------------

 \begin{frame}[standout]

Thank you! 

jkalley14@tamu.edu

 \end{frame}


%-----------------------------------------------

\appendix 


%-----------------------------------------------

\begin{frame}{Limitations}

\begin{enumerate}
\item Domestic political economy of military spending. 
\item Measurement error and missing data. 
\item Strategic alliance design
\end{enumerate}

\end{frame}


%-----------------------------------------------

\begin{frame}{Spending Growth and the Hypotheses}

\begin{figure}
	\centering
		\includegraphics[width=0.95\textwidth]{illus-arg.png}
	\label{fig:illus-arg}
\end{figure}


\end{frame}


%-----------------------------------------------

\begin{frame}{Trace plots: Major}

\begin{figure}
	\centering
		\includegraphics[width=0.95\textwidth]{beta-trace-maj.png}
\end{figure}


\end{frame}

%-----------------------------------------------

\begin{frame}{Trace plots: Non-Major}

\begin{figure}
	\centering
		\includegraphics[width=0.95\textwidth]{beta-trace-maj.png}
\end{figure}


\end{frame}


%------------------------------------------------


\begin{frame}{Alliance-Level Regression Table: Major Powers}

930 observations, with 130 alliances. 

\resizebox{.95\textwidth}{!}{
\begin{tabular}{rrrrrrr}
  \hline
 & mean & S.D. & 5\% & 95\% & n\_eff & $\hat{R}$ \\ 
  \hline
Constant & 0.038 & 0.038 & -0.025 & 0.102 & 3380.954 & 1.000 \\ 
  Latent Str. & -0.054 & 0.031 & -0.107 & -0.005 & 3278.923 & 1.000 \\ 
  Number Members & 0.000 & 0.002 & -0.003 & 0.003 & 4000.000 & 0.999 \\ 
  Democratic Membership & -0.009 & 0.033 & -0.065 & 0.042 & 4000.000 & 1.000 \\ 
  Wartime & -0.057 & 0.035 & -0.115 & -0.001 & 4000.000 & 1.001 \\ 
  Asymmetric & 0.053 & 0.035 & 0.001 & 0.115 & 2218.509 & 1.000 \\ 
  US Member & 0.002 & 0.031 & -0.051 & 0.051 & 4000.000 & 1.000 \\ 
  USSR Member & 0.023 & 0.033 & -0.028 & 0.079 & 4000.000 & 1.000 \\ 
  $\sigma$ Alliances & 0.066 & 0.029 & 0.019 & 0.117 & 599.081 & 1.007 \\ 
   \hline
\end{tabular}
}


\end{frame}

%-----------------------------------------------

\begin{frame}{Alliance-Level Regression Table: Non-Major Powers}

8,668 observations and 192 alliances. 

\resizebox{.95\textwidth}{!}{
\begin{tabular}{rrrrrrr}
  \hline
 & mean & sd & 5\% & 95\% & n\_eff & $\hat{R}$ \\ 
  \hline
Constant & -0.018 & 0.018 & -0.047 & 0.012 & 2211.374 & 1.000 \\ 
  Latent Str. & 0.026 & 0.017 & -0.002 & 0.054 & 2191.382 & 1.000 \\ 
  Number Members & 0.000 & 0.001 & -0.001 & 0.001 & 4000.000 & 1.000 \\ 
  Democratic Membership & -0.031 & 0.015 & -0.056 & -0.009 & 3213.621 & 1.000 \\ 
  Wartime & 0.041 & 0.023 & 0.002 & 0.078 & 4000.000 & 1.000 \\ 
  Asymmetric & -0.031 & 0.021 & -0.065 & 0.003 & 4000.000 & 0.999 \\ 
  US Member & 0.013 & 0.018 & -0.016 & 0.042 & 2895.419 & 1.000 \\ 
  USSR Member & 0.011 & 0.031 & -0.041 & 0.062 & 4000.000 & 1.000 \\ 
  $\sigma$ Alliances & 0.014 & 0.009 & 0.002 & 0.030 & 1254.268 & 1.001 \\ 
   \hline
\end{tabular}
}



\end{frame}


%------------------------------------------------


\begin{frame}{Priors}

4 Chains with 2,000 samples and 1,000 warmup iterations. 

\begin{table} % Create a table of priors.

 \begin{center}
\begin{tabular}{c} 
$ p(\alpha) \sim N(0, 1)$  \\
$ p(\sigma) \sim \mbox{half-}N(0, 1) $ \\
$ p(\alpha^{yr}) \sim N(0, \sigma^{yr}) $ \\ 
$ p(\sigma^{yr}) \sim N(0, 1) $ \\
$ p(\alpha^{st}) \sim N(0, \sigma^{st}) $ \\ 
$ p(\sigma^{st}) \sim \mbox{half-}N(0, 1) $ \\ 
$ p(\sigma^{all}) \sim \mbox{half-}N(0, 1) $ \\
$ p(\beta) \sim N(0, 1) $ \\
$ p(\gamma) \sim N(0, 1) $ \\ 
$ p(\nu) \sim gamma(2, 0.1)$ 
\end{tabular} 
\end{center} 
\label{tab:priors}
\end{table} 


\end{frame}


%------------------------------------------------

\begin{frame}{Details of Measurement Model}

\begin{itemize}
\item Bayesian Gaussian Copula Factor Model: for mixed data. 
\item Uses copulas to break dependence between latent factors and marginal distributions. 
\item Treats marginals as unknown and keeps them free of dependence. 
\item IMH proposal, 10,000 iteration warmup, 20,000 samples, thinned every 20 draws. 
\item Generalized double Pareto prior for the factor loading--- flexible generalized Laplace distribution with a spike at zero and heavy tails. 
\end{itemize} 


\end{frame}


%------------------------------------------------

\begin{frame}{Notable Major Power Alliances}


\begin{figure}
	\centering
		\includegraphics[width=0.95\textwidth]{non-zero-maj.png}
	\label{fig:non-zero-maj}
\end{figure}


\end{frame}

%------------------------------------------------

\begin{frame}{Notable Non-Major Power Alliances}


\begin{figure}
	\centering
		\includegraphics[width=0.95\textwidth]{non-zero-min.png}
	\label{fig:non-zero-min}
\end{figure}


\end{frame}


%------------------------------------------------

\begin{frame}{Non-Major Powers in NATO: Belgium}


\begin{figure}
	\centering
		\includegraphics[width=0.95\textwidth]{bel-agg-imp.png}
\end{figure}


\end{frame}


%------------------------------------------------

\begin{frame}{Impact of NATO on Belgium}


\begin{figure}
	\centering
		\includegraphics[width=0.95\textwidth]{bel-nato-imp.png}
\end{figure}


\end{frame}


%------------------------------------------------

\begin{frame}{Impact of EU on Belgium}


\begin{figure}
	\centering
		\includegraphics[width=0.95\textwidth]{bel-eu-imp.png}
\end{figure}


\end{frame}




%------------------------------------------------

\begin{frame}{Varying Slopes Model}

Within each of the $j$ groups of state capability, for $i$ in $1 ... n_j$: 
\begin{equation}
y_i \sim student_t(\nu_j, \alpha_j + \alpha^{st} + \alpha^{yr} +\textbf{W}_{i} \gamma  + \textbf{Z}_{ji} \lambda_{j}, \sigma_j) 
\end{equation} 

\begin{equation}
\lambda_{j} \sim N(\theta_{j}, \sigma^{all}_{j})
\end{equation} 

\begin{equation}
\theta_{j} = \alpha^{all}_{j} + \textbf{X} \beta_j
\end{equation}

I give $\beta_j$ a multivariate normal prior with prior scale $\tau$:
\begin{equation}
\beta_j \sim MVN(\mu_{\beta_j}, \Sigma_{\beta}) 
\end{equation}

\end{frame}


%------------------------------------------------

\begin{frame}{Varying Slopes Results: Scope}

\begin{figure}[htbp]
	\centering
		\includegraphics[width=0.95\textwidth]{var-slopes-scope.png}
\end{figure}

\end{frame}

%------------------------------------------------

\begin{frame}{Full Varying Slopes Results}

\begin{figure}[htbp]
	\centering
		\includegraphics[width=0.95\textwidth]{vs-res-full.png}
\end{figure}

\end{frame}


%------------------------------------------------


\begin{frame}{Single-Level Robust Regression}

\begin{figure}[htbp]
	\centering
		\includegraphics[width=0.95\textwidth]{robust-reg-coef.png}
	\label{fig:robust-reg-coef}
\end{figure}



\end{frame}



%----------------------------------------------------------------------------------------

\end{document}